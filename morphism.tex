\documentclass[UTF8]{article}
\usepackage{pagecolor,lipsum}
\usepackage{amsmath}
\usepackage{amsfonts}
\usepackage{amssymb}
\usepackage{amsthm}
\usepackage{centernot}
\usepackage{xcolor}
\usepackage{fullpage}
\usepackage[inline]{asymptote}
\newtheorem{theorem}{Theorem}
\newtheorem{defintion}{Definition}
\newtheorem{collorary}{Collorary}
\newtheorem{example}{Example}
\newtheorem{remark}{Remark}
\newtheorem{note}{Note} 
% no indentation
\setlength\parindent{0pt}

\begin{document}
\title{%
  Homomorphism and Homeomorphism \\
  \large Function, Map}
\author{XXX}
\maketitle

\section{Homomorphism}
\begin{defintion}
defintion of Homomorphism
\end{defintion}

\begin{example} 
Homomorphism is preserve structure
\end{example}

\subsection{What is function}
\begin{defintion}
defintion of function.
\end{defintion}
\begin{example} 
Group, Ring Field
\end{example}

\begin{asy}
size(5cm);
for (int n = 3; n <= 7; ++n) {
    draw(shift(2.2*n, 0) * polygon(n));
}
\end{asy}


\section{Homeomorphism}
\begin{defintion} 
defintion of Homeomorphism
\end{defintion}

\begin{example}
$ R = (\mathbb{R}, +, *)$ 
\end{example}

\begin{example}
$ G =(\mathbb{R}, \oplus)$ 
\end{example}

\end{document}
