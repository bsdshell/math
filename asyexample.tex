\documentclass{article}
\usepackage{amsmath}
\usepackage{amsfonts}
\usepackage{amssymb}
\usepackage{amsthm}
\setlength\parindent{0pt}
\usepackage[inline]{asymptote}

\newtheorem{theorem}{Theorem}
\newtheorem{defintion}{Definition}
\newtheorem{collorary}{Collorary}
\newtheorem{example}{Example}
\newtheorem{remark}{Remark}
\newtheorem{note}{Note}

\begin{document}

\begin{enumerate}
\item Latex is cool
\item Haskell is fun
    \begin{enumerate}
    \item Binary Tree
    \item quick sort is $\mathcal{O}(n\log{}n)$  
    \item merge sort is $\mathcal{O}(2^n)$  
    \item extends 
    \item merge sort is O(nlog n)
    \item quick sort is O(nlog n)
    \item Could you give me an update on the essential result? 
    \item Could you give me an update on the test result?
    \end{enumerate}
\end{enumerate} 

\[ A= \begin{bmatrix}
3 & 1\\
4 & 3 
\end{bmatrix} 
\]

\[
\left| \begin{array}{ccc}
1 & 2 & 3 \\
4 & 5 & 6 \\
9 & 0 & 9 \end{array} \right|
\] 

\[
\left| \begin{array}{ccc}
a & b & c \\
d & e & f \\
g & h & i \end{array} \right|
\] 

\begin{defintion}
What is topology?
Topology is about the shape of objects. There is no length.
Objects can be deformed to different shape(figure) according to some rules.
The typical example is donuts and coffee cup are the same since they both have
only one hole.
\end{defintion}

\begin{asy}
    settings.render=16;
    size(345.0pt,0);
    import graph3;
    currentprojection = perspective(20*dir(15,0));
    real r1=5, r0=1;
    int nu = 36, nv = 26;
    path3 crossSection = Circle(r=r0, c=(r1,0,0), normal=Y, n= nu);
    pen colorFunction(int u, real theta) {
        real z = sin(u/nu * 2pi);
        real t = (z + 1) / 2;
    return t*red + (1-t)*lightblue;
    }
    surface torus = surface(crossSection, c=(0,0,0), axis=Z, n=nv,
        angle1=90, angle2=10, color=colorFunction);
    draw(torus);
\end{asy}

\section{Introduction}
Hello World Theorem
\begin{theorem}
Any two straight lines intersect at lease one point in projective space.
This is good theorem, and it is very cool
This is good theorem, and it is very cool
\end{theorem}

\begin{theorem}
Two points defines a line
\end{theorem}

$E = MC^2$ \\
$E = MC^2$ \\
$E = MC^2$ \\

\begin{equation}
\begin{aligned}
x & = y + 1 \\
x & = z + 3 \\
x & = z + 3 \\
x & = z + 3 \\
\end{aligned}
\end{equation} 

\begin{enumerate}
\item Latex is cool
\item Haskell is fun
\begin{enumerate}
\item Binary Search Tree[BST]
\item Binary Tree
\item Binary Search Tree
\item quick sort is $\mathcal{O}(2^n)$  
\item merge sort is $\mathcal{O}(2^n)$  
\item Continuous
\item Continue
\item Consecutive
\item Contingent
\item Contingent
\item Contingent
\item The runtime is $\mathcal{O}(n\log{}n)$ 
\end{enumerate}
\end{enumerate} 

\pagebreak
What is different between quick sort and merge sort? 
\begin{enumerate}
\item Quick Sort average runtime is $\mathcal{O}(n\log{}n)$  
\item Merge sort average runtime is $\mathcal{O}(n\log{}n)$  
\item The worst runtime for quick sort is $\mathcal{O}(n^2)$  
\item The worst runtime for merge sort is $\mathcal{O}(n\log{}n)$ 
\item Memory space for QuickSort is $\mathcal{O}(1)$ 
\item Memory space for MergeSort is $\mathcal{O}(n)$
\item Quick Sort is unstable sort
\item MergeSort is stable sort
\end{enumerate} 


$\mathcal{O}(n\log{}n)$
$\mathcal{O}(n^2)$
$\mathcal{O}(2^n)$ 

\begin{theorem}
Three points are colinear if three points are on the same line.
\end{theorem}

\begin{defintion}
The sum of angles for any triangle is 180 degrees in Euclidean Space
\end{defintion}

\begin{collorary}
For all $n$ such as $n \in \mathbb{N}$ is either even or odd
\end{collorary}



\begin{note}
Negative number is true for above collorary.
\end{note}
\end{document}
