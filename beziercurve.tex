\documentclass{article}
\usepackage[tc]{titlepic}
\usepackage{xcolor}
\usepackage{graphicx}
\usepackage{tipa}
\usepackage{pagecolor,lipsum}
\usepackage{amsmath}
\usepackage{amsfonts}
\usepackage{amssymb}
\usepackage{centernot}
\usepackage{xcolor}
\definecolor{title}{RGB}{180,0,0}
\definecolor{other}{RGB}{171,0,255}
\definecolor{name}{RGB}{255,0,0}
\definecolor{phd}{RGB}{0,0,240}

\newcommand{\polyringz}[2][Z]{\mathbb{#1}\scalebox{1}[.95]{[}#2\scalebox{1}[.95]{]}}
\newcommand{\polyringr}[2][R]{\mathbb{#1}\scalebox{1}[.95]{[}#2\scalebox{1}[.95]{]}}

\begin{document}
    $\textbf{Bezier Curve}$ is used to draw smooth curve along points on a path
    A Bezier Curve goes through points called anchor points and the shape between
    anchor points is defined by so call control points\\

    $\textbf{Movement between two points using vectors}$ \\ 
    A point $\mathbf{Q}$ that moves from $\mathbf{P_1}$ to $\mathbf{P_2}$ over the time $0 \leq t \leq 1$
    can be described using position vectors. Let $\mathbf{O}$ be some fixd point, then the $\textbf{position vector}$
    $\mathbf{OQ}$ points at $\mathbf{Q}$. Let $\mathbf{P_1 P_2}$ be the vector between $\mathbf{P_1}$ and $\mathbf{P_2}$,
    then $\mathbf{OQ}$ is described by 
    \[ 
       \overrightarrow{OQ}  =  \overrightarrow{O P_1} + t\overrightarrow{P_1 P_2} \quad t \in [0, 1]
    \]
    to find the coordinates of $\overrightarrow{P_1 P_2}$, position vector are used,\\
    \[ \overrightarrow{P_1 P_2} = \overrightarrow{O P_2} - \overrightarrow{O P_1} \]
    We get\\
    \begin{align*}
       \overrightarrow{OQ} &= \overrightarrow{O P_1} + t\overrightarrow{P_1 P_2} \\
                           &= \overrightarrow{O P_1} + t (\overrightarrow{O P_2} - \overrightarrow{O P_1}) \\
                           &= (1-t)\overrightarrow{O P_1} + t \overrightarrow{O P_2}\\
    \end{align*}
    the position vector of a point has the same coordinates as the point. For this reason, you don't need to distinguish
    between points and vectors when doing computer graphic. The point $\mathbf{Q}$ can be written as  
    \[ \mathbf{Q} = (1-t) \mathbf{P_1} + t \mathbf{P_2} \quad t \in [0, 1] \]
    \textbf{Parametric equation of Bezier Curve}\\
    \textit{Linear Bezier Curve}: Two points are needed. Both are anchor points, as $\mathbf{Q_0}$ moves along the line $\mathbf{P_0}$ and $\mathbf{P_1}$ 
    it traces out a linear Bezier curve. Let t be a parameter, then the linear Bezier curve can be written as parameter curve
    \[ \mathbf{Q_0} = (1-t) \mathbf{P_0} + t \mathbf{P_1} \quad t \in [0, 1] \]
    \textit{Quadratic Bezier curve}: Three points $P_0, P_1, P_2$ are needed. $P_0, P_2$ are anchor points, $P_1$ is a control point.
    The $\mathbf{Q}$-points move along lines between the $\mathbf{P}$-points. As $R_0$ moves along a line $Q_0$ and $Q_1$, it traces out a quadratic Bezier
    curve, the three movements are described by\\
    \begin{equation}
    \begin{aligned} 
        \mathbf{Q_0} &= (1-t) \mathbf{P_0} + t\mathbf{P_1} \\
        \mathbf{Q_1} &= (1-t) \mathbf{P_1} + t\mathbf{P_2} \\
        \mathbf{R_0} &= (1-t) \mathbf{Q_0} + t\mathbf{Q_1} \\ 
        \mathbf{R_0} &= (1-t)^2 \mathbf{P_0} + 2(1-t)t \mathbf{P_1}  + t^2 \mathbf{P_2} \quad t \in [0, 1] \nonumber
    \end{aligned} 
    \end{equation} \\

    Just use $\mathbf{P}$ point to describe the movement $R_0$ we get that \\
    \[ 
        \mathbf{R_0} = (1-t)^2 \mathbf{P_0} + 2(1-t)t \mathbf{P_1}  + t^2 \mathbf{P_2} \quad t \in [0, 1]
    \]
    $\textit{Cubic Bezier Curve}$: for points $P_0$, $P_1$, $P_2$ $P_3$ are needed. $P_0$ and $P_3$ are anchor points, the other two are control points.
    Just use $\mathbf{P}$-points to describe $\mathbf{T_0}$ we get that
    \begin{equation}
    \begin{aligned} 
        \mathbf{R_0} &= (1-t) \mathbf{Q_0} + t\mathbf{Q_1} \\
        \mathbf{R_1} &= (1-t) \mathbf{Q_1} + t\mathbf{Q_2} \\
        \mathbf{S_0} &= (1-t) \mathbf{R_0} + t\mathbf{R_1} \\ 
        \mathbf{S_0} &= (1-t) ((1-t) \mathbf{Q_0} + t\mathbf{Q_1}) + t ((1-t) \mathbf{Q_1} + t\mathbf{Q_2}) \\ 
        \mathbf{S_0} &= (1-t)^3 \mathbf{P_0} + 3(1-t)^2 t\mathbf{P_1} + 3(1-t) t^{2}\mathbf{P_3} + t^{3} \mathbf{P_4} \nonumber
    \end{aligned} 
    \end{equation}
    $\textit{Quartic Bezier Curve}$: Five points $P_0$, $P_1$, $P_2$, $P_3$, $P_4$ are needed, $P_0$ and $P_4$ are anchor points, the others are control points. 
    Just use $\mathbf{P}$-points to describe $\mathbf{T_0}$ we get that
    \[ 
        \mathbf{T_0} = (1-t)^4 \mathbf{P_0} + 4(1-t)^3 t\mathbf{P_1} + 6(1-t)^2 t^{2}\mathbf{P_2} + 4(1-6) t^3 \mathbf{P_3} + t^4 \mathbf{P_4} 
    \]
\end{document}
