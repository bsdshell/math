\documentclass[UTF8]{article}
\usepackage{pagecolor,lipsum}
\usepackage{amsmath}
\usepackage{amsfonts}
\usepackage{amssymb}
\usepackage{amsthm}
\usepackage{centernot}
\usepackage{tikz}
\usepackage{xcolor}
\usepackage{fullpage}
\usepackage[inline]{asymptote}
\usepackage{listings}
\usepackage{tkz-euclide}
\usetikzlibrary{arrows,decorations.pathmorphing,backgrounds,positioning,fit,petri}
\newtheorem{theorem}{Theorem}
\newtheorem{definition}{Definition}
\newtheorem{collorary}{Collorary}
\newtheorem{example}{Example}
\newtheorem{remark}{Remark}
\newtheorem{note}{Note} 
% no indentation
\setlength\parindent{0pt}

\begin{document}
% euclideanspace euclidean_space
\section{Euclidean Space}
Let $V \in \mathbb{R}^n$ is vector space and an inner product is a function $V \times V \rightarrow \mathbb{R}$, that is 
$(x, y) \rightarrow \left< x \,, y \right>$ for $x, y \in V$, which satisfied following axioms:

\begin{equation}
\begin{aligned}
\left<ax \,, y \right> &= a \left< x \,, y \right> \\  
\left<x \,, by \right> &= b \left< x \,, y \right> \\
\left<x + y \,, z \right> &= \left< x \,, z \right> + \left< y \,, z \right> \\
\left<x\,, y + z \right> &= \left< x \,, y \right> + \left< x \,, z \right> \\
\left<x \,, y \right> &= \left<x \,, y \right> \\
\left<x \,, x \right> &> 0 \quad x \neq 0 \quad \text{positive definite}\\
\end{aligned}
\end{equation} 
\section{Construct perspective projective matrix}
Model Geometry $\rightarrow$ World Coordinates $\rightarrow$ Projective Coordinates
$\rightarrow$ Normal Device Coordinates
\[  
    \left[ \begin{array}{c} 
    x_{e} \\ 
    y_{e} \\ 
    z_{e} \\ 
    w_{e} \\ 
    \end{array} \right] 
    = \begin{bmatrix} 
    A & B & C & D\\ 
    A & B & C & D\\ 
    A & B & C & D\\ 
    A & B & C & D\\ 
    \end{bmatrix} 
    \left[ 
    \begin{array}{c} 
    x_{obj} \\ 
    y_{obj} \\ 
    z_{obj} \\ 
    w_{obj} \\ 
    \end{array} 
    \right] \\
\]

\section{Coordinate Systems and Frames}
Vector $v$ is defined as 
\[
    v = \alpha_1 e_1 + \alpha_2 e_2 + \alpha_3 e_3
\]
$\alpha_1, \alpha_2, \alpha_3$ are components of $v$ with respect to basis vector $e_1, e_2, e_3$ \\
e.g. vector 
$v = \left[ \begin{array}{c} 
                1 \\ 
                2 \\
                3
            \end{array}
            \right]
                $
can be uniquely define as following with respect to $\{e_1, e_2, e_3\}$ 
%
\begin{equation}
\begin{aligned} 
    e_0 &= \left[ \begin{array}{c} 
                1 \\ 
                0 \\
                0 
            \end{array}
            \right] \quad 
    e_1 = \left[ \begin{array}{c} 
                0 \\ 
                1 \\
                0 
            \end{array}
            \right] \quad 
    e_2 = \left[ \begin{array}{c} 
                0 \\ 
                0 \\
                1 
            \end{array}
            \right]  \\
    v &= 1e_0 + 2e_1 + 3e_2        
\end{aligned}
\end{equation} 
Vector $v$ forms a coordinate system in Vector space \\
Points representation needed a "fix" origin - reference point and basis vectors are required $\mathbf{frame}$ \\
Within a given $\mathbf{frame}$ every vector can be written uniquely 
\[
    v = \alpha_1 e_1 + \alpha_2 e_2 + \alpha_3 e_3
\]
%
Every point can be written uniquely as \\
\begin{equation}
\begin{aligned}
    p &= p_0 + \alpha_1 e_1 + \alpha_2 e_2 + \alpha_3 e_3 \\
    p &= e_1 \alpha_1  + e_2 \alpha_2 + e_3 \alpha_3 + p_0 1 \\
    p &= \left[ \begin{array}{c|c|c|c} 
    e_1 & e_2 & e_3 & p_0 
    \end{array} 
    \right] 
    \left[ \begin{array}{cc} 
    \alpha_1 \\
    \alpha_2 \\
    \alpha_3 \\
    1 
    \end{array} 
    \right] \\
    %
\end{aligned}
\end{equation} 

\section{Affine}
Affine space can be defined as $\mathbb{A}^n = \mathbf{R}^n / \text{translation}$ \\
\begin{align*}
 &\left< \mathbf{E}, \mathbf{V}, + \right> \\
 &\mathbf{E} \text{ is a set of points} \\
 & \mathbf{V} \text{ is vector space. e.g. satisfy vectors space rules.} \\
\end{align*} \\
satisfy the following rules:
any two points $p_0, p_1 \in \mathbf{E}$, there is unique vector $\vec{v} \in \mathbf{V}$ such that $\vec{v} = p_1 - p_0$ or $p_1 = p_0 + (p_1 - p_0)$
$\vec{0} = p_0 - p_0$ \\
%
\section{Point and Vector}
Point and Vector are very confusing concept since we abuse the concept and notation.
When we are talking point, an origin are needed to represent the point. e.g. \\
In two dimension Cartesian System, we assume the origin is $(0, 0)$, but we can change the origin to $(1, 2)$
Assume we choose the orgin is $(0, 0)$ and a point is $(2, 3)$ \\
Point \\
1. Point has position. \\
2. Point does not have size \\
3. Point could move to different position \\
Vector \\
1. Vector has magnitude \\
2. Vector has length \\
3. Vector does not have position \\
What are the difference betweeen Point and Vector \\
Given two point $p_0, p_1$ \\
$\vec{v} = p_0 p_1 = p_1 - p_0$
%
%
%
%
\section{Frame}
Frame is defined as a pair 
$(\mathbf{O}, \{e_1, e_2, \dots e_n\})$ consists of an origin $\mathbf{O}$(which is a point) in $\mathbf{E}$ together with a basis of three vectors $\{e_1, e_2, e_3\}$ in $\mathbf{V}$ \\
For example, the standard frame 
$(\mathbf{O} = \{0, 0, 0\}, \{e_1, e_2, e_3\})$ in $\mathbf{R}^3$
\section{Affine Combination}
%
%
\end{document}
