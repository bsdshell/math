\documentclass{article}
\usepackage[tc]{titlepic}
\usepackage{xcolor}
\usepackage{graphicx}
\usepackage{tipa}
\usepackage{pagecolor,lipsum}
\usepackage{amsmath}
\usepackage{amsfonts}
\usepackage{amssymb}
\usepackage{amsthm}
\usepackage{centernot}
\usepackage{xcolor}
\usepackage{listings}
\newtheorem{theorem}{Theorem}
\newtheorem{defintion}{Definition}
\newtheorem{collorary}{Collorary}
\newtheorem{example}{Example}
\newtheorem{remark}{Remark}
\newtheorem{note}{Note} 

\newcommand{\BB}[1]{
    \mathbb{#1}
}

\begin{document}

\section{Introduction}
The document introducts how to use Asymptotes inside Latex
Asymptotes is graphic tool that can draw 2D and 3D vector graphic.
Asymptotes code can be embedded inside Latex code
\begin{enumerate}
\item What is Latex 
\item How to write "Hello World" in Latex
    \begin{enumerate}
    \item How to draw line  
    \item How to draw Cartesian Coordinates  
    \item How to draw Curve $x \mid y$ 
    \item How to draw Curve $ \{ \frac{x + y}{a + b} \mid \frac{x+3}{y + x} \}$ 
    \item How to draw three dimensions surface 
    \item How to draw three dimensions surface 
    \item How to draw three dimensions surface 
    \item How to draw three dimensions surface 
    \end{enumerate} 
\end{enumerate}

\begin{defintion}
This is defintion what Group Theory.
What is Group?
Group is formed by one operator and operands
Given a set of objects $s \in S$ and one binary operator $\oplus$
$(s, \oplus)$ is called a group with following rules
\begin{enumerate}
\item There exists identity $1$ such as
    \begin{enumerate}
    \item $1 \oplus s = s \oplus 1 = s$ 
    \end{enumerate} 

\item There exists inverse $s'$ such as
    \begin{enumerate}
    \item $s' \oplus s = 1 \quad \forall s \in S$ 
    \end{enumerate} 

\item There satisfies Associativity 
    \begin{enumerate}
    \item  $s_1 \oplus s_2 \oplus s_3 = s_1 \oplus (s_2 \oplus s_3) \quad \forall s_1, s_2, s_3 \in S$ 
    \end{enumerate} 
\end{enumerate} 
\end{defintion}

\begin{defintion}
What is Homomorphism and Homeomorphism
\end{defintion}

\[
    \left| \begin{array}{ccc}
    a & b & c \\
    d & e & f \\
    g & h & i \end{array} \right|
\] 

\begin{theorem}
Elliptic Curve is two variables polynomial, the standard form is $y^2 = x^3 + ax + c$ 
Since $y^2 = x^3 + ax^2 + bx + c$ can be conver to $y^2 = x^3 + ax + c$
The nice think about Elliptic Curve is each three points forms a group.
e.g. If points $p_0, p_1$ the curve $C$, then you can compute the third point
from $p_0 and p_1$
e.g. $p_0 \oplus p_1 = p_2$
\end{theorem} 

\begin{theorem}
Quadratic polynomial \[ f(x) = ax^2 + bx + c  \]
if $\Delta = b^2 - 4ac$ if $\Delta > 0$ 
then $f(x)$ has two solutions

if $\Delta = b^2 - 4ac$ if $\Delta = 0$
then $f(x)$ has one solutions

if $\Delta = b^0 - 4ac$ if $\Delta < 0$
then $f(x)$ has no solutions $\quad \forall x \in \mathbb{R}$
\end{theorem} 

\begin{example}
Given a polynomial $y^2 = x^3 + 2x + 3$
\end{example} 


\begin{theorem}
Pythagorean Theorem is very important theorem in Geometry.
In a triangle, let three sides to be $a, b, c$ 
if $c^2 = a^2 + c^2$, then the triangle is right triangle.
\end{theorem} 

This is rotation matrix, $\beta$ is rotational angle
The matrix is for two dimension2, what is three dimensions matrix?
I will show your guys how to define the three dimensions matrices

\[ 
    A= \begin{bmatrix}
    \cos(\beta) & -\sin(\beta)\\
    \sin(\beta) & \cos(\beta)
    \end{bmatrix} 
\]


\[ 
    A= \begin{bmatrix}
    1 & 2 & 3\\
    4 & 5 & 6\\
    7 & 8 & 9
    \end{bmatrix} 
\]

\[ A= \begin{bmatrix}
1 & 2 & 3\\
4 & 5 & 6\\
7 & 8 & 9
\end{bmatrix} 
 \]

\[ A= \begin{bmatrix}
1 & 0\\
0 & 1
\end{bmatrix} 
 \]

\[ 
    A= \begin{bmatrix}
    1 & 0 & 0\\
    0 & 1 & 0\\
    0 & 0 & 1
    \end{bmatrix} 
 \]

\end{document}
