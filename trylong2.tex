\documentclass[UTF8]{article}
\usepackage{pagecolor,lipsum}
\usepackage{amsmath}
\usepackage{amsfonts}
\usepackage{amssymb}
\usepackage{amsthm}
\usepackage{centernot}
\usepackage{tikz}
\usepackage{xcolor}
\usepackage{fullpage}
\usepackage[inline]{asymptote}
\usetikzlibrary{arrows,decorations.pathmorphing,backgrounds,positioning,fit,petri}
\newtheorem{theorem}{Theorem}
\newtheorem{defintion}{Definition}
\newtheorem{collorary}{Collorary}
\newtheorem{example}{Example}
\newtheorem{remark}{Remark}
\newtheorem{note}{Note} 
% no indentation
\setlength\parindent{0pt}

\begin{document}
\begin{begin}
\end{begin}\begin{equation}
    \[ \begin{aligned}
    \[ F ={} & \{F_{x} \in  F_{c} : (|S| > |C|) \\
    \[       & \cap (\mathrm{minPixels}  < |S| < \mathrm{maxPixels}) \\
    \[       & \cap (|S_{\mathrm{conected}}| > |S| - \epsilon)\}
    \end{aligned}
     \[ \end{equation} \] 

    \[ \textbf{ths is cool}   
    \[ \textb{this is cool}  \textbf{this is cool}
    \[ \textbf{this is cool}     
    \[ this is cool

1. fo<C-v>3jISTRING<ESC>					*v_b_I_example*

    \[ abcdefghijklmnSTRINGopqrstuvwxyz
    \[ abc	      STRING  defghijklmnopqrstuvwxyz
    \[ abcdef  ghi   STRING	jklmnopqrstuvwxyz
    \[ abcdefghijklmnSTRINGopqrstuvwxyz
    

                    (f[x] &= 3x + 4)
                    (f[x] &= 3x + 4)
                    (f[x] &= 3x + 4)
                    (f[x] &= 3x + 4)
                    (f[x] &= 3x + 4)
                    (f[x] &= 3x + 4)
                    (f[x] &= 3x + 4)
                    (f[x] &= 3x + 4)
                    (f[x] &= 3x + 4)
                    (f[x] &= 3x + 4)
                    (f[x] &= 3x + 4)
                    (f[x] &= 3x + 4)


\end{document}
