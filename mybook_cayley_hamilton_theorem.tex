\documentclass[10pt]{article}
\usepackage{tipa}
\usepackage{pagecolor,lipsum}
\usepackage{amsmath}
\usepackage{amsfonts}
\usepackage{amssymb}
\usepackage{centernot}
\usepackage{xcolor}
\usepackage{graphicx}
\usepackage{svg}
\newcommand{\polyringz}[2][Z]{\mathbb{#1}\scalebox{1}[.95]{[}#2\scalebox{1}[.95]{]}}
\newcommand{\polyringr}[2][R]{\mathbb{#1}\scalebox{1}[.95]{[}#2\scalebox{1}[.95]{]}}
\newtheorem{propo}{Proposition}[section]
\newtheorem{defin}{Definition}[section]
\newtheorem{theorem}{Theorem}[section]

\begin{document}
    \[ \textbf{Similar Matrix} \]
\begin{defin}
    For invertable matrix $\mathbf{A}$, if there exist matrix $\mathbf{P}$ such that
    \[ \mathbf{A}  = \mathbf{P} \mathbf{B} \mathbf{P}^{-1} \]
    then $\mathbf{A}$ is similar to $\mathbf{B}$
\end{defin}

\begin{propo}
Let $\mathbf{A}$ be an $n \times n$ matrix, $\mathbf{P}$ is non-singular $n \times n$ matrix, and
$\mathbf{B} = \mathbf{P} \mathbf{A} \mathbf{P}^{-1}$. Then the matrices have the same characteristic polynomials.
\end{propo}

Proof. The characteristic polynomials of $\mathbf{A}$ and $\mathbf{B}$ are given as 
    \begin{equation}
    \begin{aligned}
     \det (\mathbf{A} - \lambda \mathbf{I}) & = \det (\mathbf{P} \mathbf{B} \mathbf{P}^{-1} - \lambda \mathbf{I}) \\
     & = \det (\mathbf{P} \mathbf{B} \mathbf{P}^{-1} - \lambda \mathbf{P} \mathbf{P}^{-1}) \\ 
     & = \det (\mathbf{P} \mathbf{B} \mathbf{P}^{-1} - \mathbf{P} \lambda \mathbf{P}^{-1}) \\
     & = \det (\mathbf{P} \mathbf{B} \mathbf{P}^{-1} - \mathbf{P} \lambda \mathbf{I} \mathbf{P}^{-1})  \\
     & = \det (\mathbf{P} (\mathbf{B} - \lambda \mathbf{I}) \mathbf{P}^{-1}) \\
     & = \det (\mathbf{P}) \det(\mathbf{B} - \lambda \mathbf{I}) \det (\mathbf{P}^{-1}) \\
     & = \det(\mathbf{B} - \lambda \mathbf{I}) \nonumber\\
    \end{aligned}
    \end{equation}
\[ \textbf{Cayley-Hamilton Theorem} \]

\begin{theorem}
Let $\mathbf{A}$ be an n $\times$ n matrix, and let $p_{A}(\lambda) = \det(\mathbf{A} - \lambda \mathbf{I})$ be the corresponding characteristic polynomial. 
Then $p_{\mathbf{A}}(\mathbf{A}) = \mathbf{0}$
\end{theorem}

\pagebreak
\[ \textbf{Find the characteristic polynomial of given matrix} \] 
\begin{equation} 
\begin{aligned}
    A= \begin{bmatrix}
    1 & 2 & 3\\
    0 & -2 & 1\\
    1 & 1 & 0 
    \end{bmatrix}  \nonumber
\end{aligned}
\end{equation}
\begin{equation} 
\begin{aligned}
    & \det \left( \begin{bmatrix}
    1 & 2 & 3\\
    0 & -2 & 1\\
    1 & 1 & 0 
    \end{bmatrix} - 
    \lambda
    \begin{bmatrix}
    1 & 0 & 0\\
    0 & 1 & 0\\
    0 & 0 & 1 
    \end{bmatrix} \right) \\  
    \newline 
    & \det \left( \begin{bmatrix}
    1 & 2 & 3\\
    4 & 5 & 6\\
    7 & 8 & 9 
    \end{bmatrix} - 
    \begin{bmatrix}
    \lambda & 0 & 0\\
    0 & \lambda & 0\\
    0 & 0 & \lambda 
    \end{bmatrix} \right)   \\  
    \newline 
    & \det \begin{bmatrix}
    1 - \lambda & 2 & 3\\
    4 & 5 - \lambda & 6\\
    7 & 8 & 9 - \lambda  
    \end{bmatrix}  \nonumber \\ 
\end{aligned}
\end{equation}

\begin{equation} 
\begin{aligned}
    p_{\mathbf{A}}(\lambda) = 
    & (-1)^{1+1} (1 - \lambda) 
    \begin{vmatrix}
    5 - \lambda & 6 \\
    8 & 9 - \lambda  
    \end{vmatrix} +  \\
    & (-1)^{1+2} 4 
    \begin{vmatrix}
    2 & 3 \\ 
    8 & 9 - \lambda 
    \end{vmatrix} +  \\
    & (-1)^{1+3} 7 
    \begin{vmatrix}
    2 & 3 \\
    5 - \lambda & 6 
    \end{vmatrix} \\
    p_{\mathbf{A}}(\lambda) = 
    & (-1)^{1+1} (1 - \lambda)(5- \lambda)(9-\lambda) - 48 + \\ 
    & (-1)^{1+2} 4 \times 2 (9-\lambda) - 24 + \\ 
    & (-1)^{1+3} 7 \times 12 - 3 (15 - 3 \lambda) \\ 
    p_{\mathbf{A}}(\lambda) = 
    & (1 - \lambda)(5 - \lambda)(9 - \lambda) - 48 + \\ 
    & -8(9 - \lambda) - 24 +  \\
    & 84 - (15 - 3 \lambda) \\
    p_{\mathbf{A}}(\lambda) = 
    & (9 - \lambda) \left[ (1 - \lambda) (5 -\lambda) - 8 \right] - (15 - 3 \lambda) + 12 \\ 
    p_{\mathbf{A}}(\lambda) = 
    & - \lambda^{3} - \lambda^{2} + 6 \lambda + 7  \\ 
    p_{\mathbf{A}}(\mathbf{A}) = 
    & - \mathbf{A}^{3} - \mathbf{A}^{2} + 6 \mathbf{A} + 7\mathbf{I}  \nonumber \\ 
\end{aligned}
\end{equation}

\begin{equation} 
\begin{aligned}
    7A^{0} = 
    \begin{bmatrix}
    7 & 0 & 0\\
    0 & 7 & 0\\
    0 & 0 & 7 
    \end{bmatrix}
    6A = 
    \begin{bmatrix}
    6 & 12 & 3\\
    0 & -2 & 1\\
    6 & 6  & 0
    \end{bmatrix}  
    A^{2} = 
    \begin{bmatrix}
    4 & 1  & 5\\
    1 & 5 & -2\\
    1 & 0  & 4 
    \end{bmatrix}  
    A^{3} = 
    \begin{bmatrix}
    9 & 11  & 13\\
    -1 & -10 & 8\\
    5 & 6  & 3 
    \end{bmatrix} \nonumber \\
\end{aligned}
\end{equation}
    \[ p_{\mathbf{A}}(\mathbf{A}) = - \mathbf{A}^{3} - \mathbf{A}^{2} + 6\mathbf{A} + 7\mathbf{I} = \mathbf{0} \]
\end{document} 
