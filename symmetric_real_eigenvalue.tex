\documentclass{article}
\usepackage[tc]{titlepic}
\usepackage{xcolor}
\usepackage{graphicx}
\usepackage{tipa}
<<<<<<< HEAD
\usepackage{fullpage}
=======
>>>>>>> 0a4c7280fb3137ee41b70ae5da8fc8598d63c41d
\usepackage{pagecolor,lipsum}
\usepackage{amsmath,amsfonts}
\DeclareMathOperator{\SPAN}{Span}
\usepackage{amssymb}
\usepackage{amsthm}
\usepackage{centernot}
\usepackage{xcolor}
\usepackage{listings}
<<<<<<< HEAD
=======
\usepackage{fullpage}
>>>>>>> 0a4c7280fb3137ee41b70ae5da8fc8598d63c41d
\newenvironment{solution}
 {\begin{proof}[Solution]}
 {\end{proof}}
%
\newtheorem{theorem}{Theorem}
\newtheorem{definition}{Definition}
\newtheorem{collorary}{Collorary}
\newtheorem{example}{Example}
\newtheorem{remark}{Remark}
\newtheorem{note}{Note} 
%
\begin{document}
<<<<<<< HEAD
%
=======
\section{Given real a symmetric matrix $A$, prove that all the eigenvalues of $A$ are real numbers}
<<<<<<< HEAD
=======
\begin{example}
>>>>>>> 0a4c7280fb3137ee41b70ae5da8fc8598d63c41d
>>>>>>> 5239b71aa24969db1be94829f21d4c10ecead826
\begin{definition}
Given an real  $n \times n$ matrix $A$, $\lambda \in \mathbb{C} \quad  \vec{v} \in \mathbb{R}^{n}$ if 
      \[ \mathbf{A} \vec{v} = \lambda \vec{v} \]   
<<<<<<< HEAD
then $\lambda$ is the \textbf{eigenvalue} and $\vec{v}$ is the \textbf{eigenvector} such as $\vec{v} \ne \vec{0}$ 
=======
then $\lambda$ is the eigenvalue and $\vec{v}$ is the eigenvector such as $\vec{v} \ne \vec{0}$ 
<<<<<<< HEAD
>>>>>>> 5239b71aa24969db1be94829f21d4c10ecead826
\end{definition}
%
\begin{example}
\textbf{Find the eigenvalue and eigenvector of the following matrix}
=======
\end{definition} \\
\begin{example}
%
$\textbf{Find the eigenvalue and eigenvector of the following given matrix}$
>>>>>>> 0a4c7280fb3137ee41b70ae5da8fc8598d63c41d
\[  \mathbf{A}_{2\times2}(\mathbb{R})= 
    \begin{bmatrix}
    1 & 2\\
    5 & 4
    \end{bmatrix}  \]
\end{example}
<<<<<<< HEAD
\begin{align*}
\intertext{From the definition, we have $\lambda \in \mathbb{C}$ and $\vec{v} \in \mathbb{R}^{2}$}
=======
<<<<<<< HEAD
%
\begin{align*}
\intertext{From the definition, we have $\lambda \in \mathbb{C}$ and $\vec{v} \in \mathbb{R}^{n}$}
=======
from the definition, we have $\lambda \in \mathbb{C}$ and $\vec{v} \in \mathbb{R}^{n}$
%
\begin{align*}
>>>>>>> 0a4c7280fb3137ee41b70ae5da8fc8598d63c41d
>>>>>>> 5239b71aa24969db1be94829f21d4c10ecead826
    \det \mathbf{A} - \lambda \vec{I} = 0  \\
    \det\left(\begin{bmatrix}
    1 & 2\\
    5 & 4
    \end{bmatrix} - \lambda 
    \begin{bmatrix}
    1 & 0\\
    0 & 1
    \end{bmatrix} \right)  = 0  \\
    \det\left(\begin{bmatrix}
    1 - \lambda & 2\\
    5 & 4 - \lambda
    \end{bmatrix} \right)= 0 \\
    (1-\lambda)(4-\lambda)-2 \times 5  = 0 \\
    \lambda^2 - 5\lambda + 4 - 10  = 0 \\
    \lambda^2 - 5\lambda - 6  = 0 \\
    (\lambda -6) (\lambda + 1) = 0 \\ 
    \Rightarrow \lambda = 6 \mbox{ or }  \lambda = -1 
\end{align*}
Let $\lambda = 6$ and 
$\vec{v} =\left[ \begin{array}{c} 
        x \\
        y 
        \end{array} 
        \right] $ 
we have following
\begin{align*}
        \begin{bmatrix}
        1 & 2 \\
        5 & 4  
        \end{bmatrix} 
        \left[ \begin{array}{c} 
        x \\
        y 
        \end{array} 
        \right]  &= 
        6\left[ \begin{array}{c} 
        x \\
        y 
        \end{array} \right] \\
%        \begin{bmatrix}
%        1 & 2 \\
%        5 & 4  
%        \end{bmatrix} 
%        \left[ \begin{array}{c} 
%        x \\
%        y 
%        \end{array} 
%        \right] - 6  
%        \left[ \begin{array}{c} 
%        x \\
%        y 
%        \end{array} \right] &= \vec{0} \\
%        \left[ \begin{array}{c} 
%        x + 2y \\
%        5x + 4y 
%        \end{array} 
%        \right] - \left[ \begin{array}{c}   
%        6x \\
%        6y
%        \end{array} \right] &= \vec{0} \\
        \left[ \begin{array}{c} 
        x + 2y -6x\\
        5x + 4y - 6y 
        \end{array} 
        \right] &= \vec{0} \\
        \left[ \begin{array}{c} 
        -5x + 2y\\
        5x - 2y 
        \end{array} 
        \right] &= \vec{0} \\
%        \begin{bmatrix}
%        -5 & 2 \\
%        5 & -2  
%        \end{bmatrix} 
%        \left[ \begin{array}{c}   
%        x \\
%        y
%        \end{array} \right] \\
        \left[ \begin{array}{c}   
        x \\
        y
        \end{array} \right] &=  
        \left[ \begin{array}{c}   
        5 \\
        2 
        \end{array} \right] \\ 
        \mathit{E}_{\lambda=6} &= \SPAN \left\{
            \left[ \begin{array}{c}   
            5 \\
            2 
            \end{array} \right]
        \right\}
\end{align*}
Let $\lambda = -1$ we have following
\begin{align*}
        \begin{bmatrix}
        1 & 2 \\
        5 & 4  
        \end{bmatrix} 
        %
        \left[ \begin{array}{c} 
        x \\
        y 
        \end{array} 
        \right]  &= 
        -1\left[ \begin{array}{c} 
        x \\
        y 
        \end{array} \right] \\
        %
        \left[ \begin{array}{c} 
        x + 2y + x\\
        5x + 4y + y 
        \end{array} 
        \right] &= \vec{0} \\
        \left[ \begin{array}{c} 
        2x + 2y \\
        5x + 5y 
        \end{array} 
        \right] &= \vec{0} \\
        \left[ \begin{array}{c} 
        x \\
        y 
        \end{array} 
        \right] &= 
        \left[\begin{array}{c} 
        1 \\
        -1 
        \end{array} 
        \right] \\ 
        \mathit{E}_{\lambda=-1} &= \SPAN \left\{
            \left[ \begin{array}{c}   
            1 \\
            -1 
            \end{array} \right]
        \right\}
\end{align*}
<<<<<<< HEAD

%
\section{Given real a symmetric matrix $A$, prove that all the eigenvalues of $A$ are real numbers}
=======
<<<<<<< HEAD
>>>>>>> 5239b71aa24969db1be94829f21d4c10ecead826
\begin{proof}
Let $A$ is real and symmetric matrix, prove all the eigenvalues of $A$ are real,
We have following
   \[ A = A^{T} \]  
Let $\lambda \in \mathbb{C}$ to be eigenvalue, $\vec{v} \in \mathbb{R}^n$ to be the eigenvector of $A$ such as $\vec{v} \ne \vec{0}$. From the definition, we have following
    \begin{align*}
        A \vec{v} &= \lambda \vec{v} \\ 
        (A \vec{v})^{\ast}   &= (\lambda \vec{v})^{\ast} \\   
        (\vec{v})^{\ast} A^{\ast} &= (\vec{v})^{\ast} \lambda^{\ast} \\     
\intertext{Since $A$ is real and $\vec{v} \in \mathbb{R}^n$}
        (\vec{v})^{T} A &= (\vec{v})^{T} \lambda^{\ast} \qquad \because (\vec{v})^{\ast} = (\vec{v})^T \mbox{ and } A = A^{\ast}    
\intertext{Multiple both side by $\vec{v}$}
        (\vec{v})^{T} A \vec{v} &= (\vec{v})^{T} \lambda^{\ast} \vec{v} \\   
        \left< \vec{v} \,, A \vec{v} \right> &= \left< \vec{v} \,, \lambda^{\ast} \vec{v} \right> \\  
        \left< \vec{v} \,, \lambda \vec{v} \right> &= \left< \vec{v} \,, \lambda^{\ast} \vec{v} \right>\quad \because A \vec{v} = \lambda \vec{v} \\
        \overline{\lambda} \left< \vec{u} \,, \vec{u} \right> &= \lambda \left< \vec{u} \,, \vec{u} \right> \quad \because \lambda^{\ast} = \overline{\lambda} \mbox{ and } \left< \vec{v} \,, a\vec{u} \right> = \overline{a} \left< \vec{v} \,, \vec{u} \right> \\ 
        \Rightarrow \overline{\lambda} &= \lambda \\
        \Rightarrow \lambda & \in \mathbb{R}
    \end{align*}
\end{proof}
\end{document}
=======
\end{document}


>>>>>>> 0a4c7280fb3137ee41b70ae5da8fc8598d63c41d
