\documentclass[UTF8]{article}
\usepackage{pagecolor,lipsum}
\usepackage{amsmath}
\usepackage{amsfonts}
\usepackage{amssymb}
\usepackage{amsthm}
\usepackage{centernot}
\usepackage{tikz}
\usepackage{xcolor}
\usepackage{fullpage}
\usepackage[inline]{asymptote}
\usepackage{listings}
\usetikzlibrary{arrows,decorations.pathmorphing,backgrounds,positioning,fit,petri}
\newtheorem{theorem}{Theorem}
\newtheorem{defintion}{Definition}
\newtheorem{collorary}{Collorary}
\newtheorem{example}{Example}
\newtheorem{remark}{Remark}
\newtheorem{note}{Note} 
% no indentation
\setlength\parindent{0pt}
\begin{document}

\begin{lstlisting}[
mathescape,
columns=fullflexible,
basicstyle=\fontfamily{lmvtt}\selectfont,
]
Out objective for today is to show you how to use hightlight, selection and substitute
text     => $\color{red}{[}$nice$\color{red}{]}$
nice text=> $\color{red}{[}$nice nice$\color{red}{]}$

1. select text
2. press <Shift-v>
3. type  s/\%V\w\+/[\0]/gc  Enter
    what is \%V here? \%V matches inside the Visual area. If \%V is not used, the whole line is used to match
    \w\+ matches one or more [0-9] or [a-z] or [A-Z] or underscore, e.g. it match my_text123 
    \0 is the whole match
\end{lstlisting} 

\end{document}
