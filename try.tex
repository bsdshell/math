\documentclass{article}
\usepackage[tc]{titlepic}
\usepackage{xcolor}
\usepackage{graphicx}
\usepackage{tipa}
\usepackage{fullpage}
\usepackage{pagecolor,lipsum}
\usepackage{amsmath}
\usepackage{amsfonts}
\usepackage{amssymb}
\usepackage{centernot}
\usepackage{xcolor}
\usepackage{listings}

% no indentation
\setlength\parindent{0pt}

\newtheorem{theorem}{Theorem}
\newtheorem{defintion}{Definition}
\newtheorem{collorary}{Collorary}
\newtheorem{example}{Example}
\newtheorem{remark}{Remark}
\newtheorem{note}{Note} 

\begin{document}
\vspace*{0.1cm}
\large
\begin{center}
\textbf{Save your hundred of key strokes with Vim 8.0 feature}
\end{center}
\vspace*{0.2cm}

A few months ago, Vim 8.0 is released and it was a major release in last ten years. It is pretty significant. There are many new features are added in the release, one of them is called \textbf{timer} which is like a stopwatch in your phone. 

Technically specking,
\textbf{timer} is similar to event handler in $C\#$ or event listener in Java, except that it is much easier to use. 
\begin{itemize}
 \item A function is subscribed in event queue. 
 \item The function is trigged by a specific event.  
 \item[] \textbf{Another Analogy}
 \item You \textbf{join} the line to wait for ordering you Caramel Macchiato in Starbucks
 \item You \textbf{order} your Caramel Macchiato in a specific time. 
\end{itemize}


There are two important functions for timer.
\begin{itemize}
\item timer{\_}start() which starts the timer and call a function in specific time
\item timer{\_}stop() which stops the timer immediately
\end{itemize}

\begin{example}
If you want to write a function to save your current buffer in every 2 seconds,
you can do it in one line with Vim script.  
\begin{verbatim}
    func! SaveFile(timer)
      silent! :w!
    endfunc
\end{verbatim} 
\end{example}
\textbf{timer} $\quad$ A variable is initialized by timer{\_}start() \\
\textbf{silent} $\quad$ Execute silently. Normal message will not be given or added to the message history. when [!] is added, error message will also be skipped. \\
darkGrayColor
lightGrayColor
whiteColor
grayColor
redColor
greenColor
blueColor
cyanColor
yellowColor
magentaColor
orangeColor
purpleColor
brownColor
clearColor 

            however - The sport event, however, continued despite the weather. [no a compound sentense] 
            Binary Search Tree[BST] 
            

The \textbf{SaveFile} function is surprisedly simple but extremely useful because you don't need to constantly press a key to save your file when you edit a file.

\begin{verbatim}
    let timer = timer_start(2000, 'SaveFile',{'repeat':-1})
    func! SaveFile(timer)
      silent! :w!
    endfunc
\end{verbatim} 

enjoy your Vim 8 feature!

\pagebreak
\section{There is, are}
If you want to speak about the existence of a person or a thing, you should use 
$\textbf{there is/are}$

\begin{itemize}
\item[] There is man who know you 
\item[] there is many pupils in the school  
\item[] is there any exit to the street    
\item[] there are three boys in the family   
\item[] there is one table and four chair in this room  
\item[] there is one table and four chair in that room  
\item Simple past  
\item[] there was a car here, but it went away 
\item[] there were many pedestrians in the street at the time
\item Simple future
\item[] There will be a scandal if people know that
\item Future "Going to" 
\item[] is there going to be strike tomorrow
\item Present perfect
\item[] There has been a lots of victims in the accident
\item Modal Auxiliaries:Can/Can't,physical or intellectual ability/inability/impossibility
\item[] There can only be one explanation
\item[] There can not be more people in this small room
\item[] There can not be more people in this small room
\item May : Eventually/permission
\item[] There may be snow tomorrow
\item[] There may not be a mistake here
\item Must : obligation, near-certitude / interdiction
\item[] Must there be two policemen with him?
\item[] There must not be any kid here
\item Should : advice / suggestion
\item[] There should be traffic lights in the crossroads
\item[] There should not be as many fans near the singer
\item $\color{red}{\textbf{When there is no matches return a empty list}}$
\end{itemize}



\end{document} 

