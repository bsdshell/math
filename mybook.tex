\documentclass[10pt]{article}
\usepackage[tc]{titlepic}
\usepackage{xcolor}
\usepackage{graphicx}
\usepackage{tipa}
\usepackage{pagecolor,lipsum}
\usepackage{centernot}
\usepackage{xcolor}
\usepackage{fullpage}
\usepackage{amssymb}
\usepackage{amsthm}
\usepackage{amsmath,amsfonts}
\usepackage[inline]{asymptote}
\DeclareMathOperator{\SPAN}{Span}
\newtheorem{theorem}{Theorem}
\newtheorem{definition}{Definition}
\newtheorem{collorary}{Collorary}
\newtheorem{example}{Example}
\newtheorem{remark}{Remark}
\newtheorem{note}{Note} 
\newenvironment{solution}
 {\begin{proof}[Solution]}
 {\end{proof}}
\setlength\parindent{0pt}
\definecolor{title}{RGB}{180,0,0}
\definecolor{other}{RGB}{171,0,255}
\definecolor{name}{RGB}{255,0,0}
\definecolor{phd}{RGB}{0,0,240}

\newcommand{\polyringz}[2][Z]{\mathbb{#1}\scalebox{1}[.95]{[}#2\scalebox{1}[.95]{]}}
\newcommand{\polyringr}[2][R]{\mathbb{#1}\scalebox{1}[.95]{[}#2\scalebox{1}[.95]{]}}

\begin{document}
\title{\bfseries {\sc\textcolor{title}{Algebra, Geometry, Topology and Algorithm}}}
\author{\textcolor{other}{Thesis  submitted to} \\[5pt]
\emph{\textcolor{other}{unknown University}}\\[2cm]
 \textcolor{other}{in partial fulfilment my dream} \\[2cm]
\textsc{\Large{\textcolor{phd}{Doctor of Philosophy}}} \\[5pt]
  \textcolor{other}{in some crazy subject} \vspace{0.4cm} \\[1in]
  \textcolor{other}{By}\\[5pt] {\Large \sc \textcolor{name}{XXX}}
   \vspace{2cm}
\titlepic{\includegraphics[width=0.19\textwidth]{example-image-a}\\[5pt]
\textcolor{other}{Department of Subject}\\[5pt]
 \textcolor{other}{Address line -- 2}\\[5pt]
 \textcolor{other}{Address line -- 4}\\
 \vfill
 \textcolor{other}{April 2016}}}
\date{}
\maketitle

\section{Introduction}
The Latex doc will have many random stuff on it, there is no particular content or theme on the document.
I will write whatever I would like to write based on my mood. The main objective of the document is to 
help me to learn some Latex. Latex is a program language used for typesetting technical documents. 
Latex is free program package created in 1985 by the American computer scientist Leslis Lamport as an addition
to the Tex typesetting system. Latex is created to make it easier to product general-purpose books and articles within Tex
\section{Power series}
\begin{proof}
$\textbf{The power series } exp(A) = \sum_{k=0}^{n} \frac{A^{k}}{k!} \textbf{ is convegent when A is nxn matrix}$ 
\begin{align*}
    \begin{split}
    \exp(X) &= \sum_{k=0}^{n} \frac{X^{k}}{k!}\\
    \sin(X) &= \sum_{k=0}^{n} (-1)^{k}\frac{X^{2k}}{2k!}\\
    \cos(X) &= \sum_{k=0}^{n} (-1)^{k}\frac{X^{2k+1}}{(2k+1)!}\\
    \cosh(X) &= \sum_{k=0}^{n} \frac{X^{2k}}{(2k)!}\\
    \sinh(X) &= \sum_{k=0}^{n} \frac{X^{2k+1}}{(2k+1)!}\\
    \text{Let } X = \begin{bmatrix}
        0 & -\beta \\
        \beta & 0 
        \end{bmatrix} \\
    % -- not working here for \intertext{}
    \exp(X) &= 
        \begin{bmatrix}
        1 & 0 \\
        0 & 1 
        \end{bmatrix}
        +
        \begin{bmatrix}
        0 & -\beta \\
        \beta & 0 
        \end{bmatrix}
        +
        \frac{1}{2!}
        \begin{bmatrix}
        -\beta^{2} & 0 \\
        0 & \beta^{2} 
        \end{bmatrix} \\
        &\quad+
        \frac{1}{3!}
        \begin{bmatrix}
        0 & \beta^{3} \\
        -\beta^{3} & 0 
        \end{bmatrix}
        +
        \frac{1}{4!}
        \begin{bmatrix}
        \beta^{4} & 0 \\
         0 & \beta^{4} 
        \end{bmatrix}
        +
        \frac{1}{5!}
        \begin{bmatrix}
        0 & -\beta^{5}\\
        \beta^{5} & 0 
        \end{bmatrix} \\
        %
        \exp(X) &= 
        \begin{bmatrix}
        1 + \frac{1}{2!}(-\beta^{2}) + \frac{1}{4!}\beta^{4} + ... & -\beta + \frac{1}{3!}\beta^{3} + ...\\
        \beta + \frac{1}{3!}-\beta^{3} + ... & 1 + \frac{1}{2!}\beta^{2} + \frac{1}{4!}\beta^{4} + ...
        \end{bmatrix} \\
        \exp(X) &=
        \begin{bmatrix}
        \sin(\beta) & -\cos(\beta)\\
        \cos(\beta) & \sin(\beta)
        \end{bmatrix}
     \end{split}
\end{align*}
\end{proof}
\pagebreak
\section{Given real a symmetric matrix $A$, prove that all the eigenvalues of $A$ are real numbers}
\begin{definition}
Given an real  $n \times n$ matrix $A$, $\lambda \in \mathbb{K} \quad  \vec{v} \in \mathbb{R}^{n}$ if 
      \[ \mathbf{A} \vec{v} = \lambda \vec{v} \]   
then $\lambda$ is the eigenvalue and $\vec{v}$ is the eigenvector such as $\vec{v} \ne \vec{0}$ 
\end{definition}
\begin{example}
\textbf{Find the eigenvalue and eigenvector of the following matrix}
\[  \mathbf{A}_{2\times2}(\mathbb{R})= 
    \begin{bmatrix}
    1 & 2\\
    5 & 4
    \end{bmatrix}  \]
\end{example}
%
\begin{align*}
\intertext{From the definition, we have $\lambda \in \mathbb{C}$ and $\vec{v} \in \mathbb{R}^{n}$}
    \det \mathbf{A} - \lambda \vec{I} = 0  \\
    \det\left(\begin{bmatrix}
    1 & 2\\
    5 & 4
    \end{bmatrix} - \lambda 
    \begin{bmatrix}
    1 & 0\\
    0 & 1
    \end{bmatrix} \right)  = 0  \\
    \det\left(\begin{bmatrix}
    1 - \lambda & 2\\
    5 & 4 - \lambda
    \end{bmatrix} \right)= 0 \\
    (1-\lambda)(4-\lambda)-2 \times 5  = 0 \\
    \lambda^2 - 5\lambda + 4 - 10  = 0 \\
    \lambda^2 - 5\lambda - 6  = 0 \\
    (\lambda -6) (\lambda + 1) = 0 \\ 
    \Rightarrow \lambda = 6 \mbox{ or }  \lambda = -1 
\end{align*}
Let $\lambda = 6$ and 
$\vec{v} =\left[ \begin{array}{c} 
        x \\
        y 
        \end{array} 
        \right] $ 
we have following
\begin{align*}
        \begin{bmatrix}
        1 & 2 \\
        5 & 4  
        \end{bmatrix} 
        \left[ \begin{array}{c} 
        x \\
        y 
        \end{array} 
        \right]  &= 
        6\left[ \begin{array}{c} 
        x \\
        y 
        \end{array} \right] \\
        \left[ \begin{array}{c} 
        x + 2y -6x\\
        5x + 4y - 6y 
        \end{array} 
        \right] &= \vec{0} \\
        \left[ \begin{array}{c} 
        -5x + 2y\\
        5x - 2y 
        \end{array} 
        \right] &= \vec{0} \\
        \left[ \begin{array}{c}   
        x \\
        y
        \end{array} \right] &=  
        \left[ \begin{array}{c}   
        5 \\
        2 
        \end{array} \right] \\ 
        \mathit{E}_{\lambda=6} &= \SPAN \left\{
            \left[ \begin{array}{c}   
            5 \\
            2 
            \end{array} \right]
        \right\}
\end{align*}
Let $\lambda = -1$ we have following
\begin{align*}
        \begin{bmatrix}
        1 & 2 \\
        5 & 4  
        \end{bmatrix} 
        %
        \left[ \begin{array}{c} 
        x \\
        y 
        \end{array} 
        \right]  &= 
        -1\left[ \begin{array}{c} 
        x \\
        y 
        \end{array} \right] \\
        %
        \left[ \begin{array}{c} 
        x + 2y + x\\
        5x + 4y + y 
        \end{array} 
        \right] &= \vec{0} \\
        \left[ \begin{array}{c} 
        2x + 2y \\
        5x + 5y 
        \end{array} 
        \right] &= \vec{0} \\
        \left[ \begin{array}{c} 
        x \\
        y 
        \end{array} 
        \right] &= 
        \left[\begin{array}{c} 
        1 \\
        -1 
        \end{array} 
        \right] \\ 
        \mathit{E}_{\lambda=-1} &= \SPAN \left\{
            \left[ \begin{array}{c}   
            1 \\
            -1 
            \end{array} \right]
        \right\}
\end{align*}
\begin{proof}
Let $A$ is real and symmetric matrix, prove all the eigenvalues of $A$ are real,
We have following
   \[ A = A^{T} \]  
Let $\lambda \in \mathbb{C}$ to be eigenvalue, $\vec{v} \in \mathbb{R}^n$ to be the eigenvector of $A$ such as $\vec{v} \ne \vec{0}$. From the definition, we have following
    \begin{align*}
        A \vec{v} &= \lambda \vec{v} \\ 
        (A \vec{v})^{\ast}   &= (\lambda \vec{v})^{\ast} \\   
        (\vec{v})^{\ast} A^{\ast} &= (\vec{v})^{\ast} \lambda^{\ast} \\     
\intertext{Since $A$ is real and $\vec{v} \in \mathbb{R}^n$}
        (\vec{v})^{T} A &= (\vec{v})^{T} \lambda^{\ast} \qquad \because (\vec{v})^{\ast} = (\vec{v})^T \mbox{ and } A = A^{\ast}    
\intertext{Multiple both side by $\vec{v}$}
        (\vec{v})^{T} A \vec{v} &= (\vec{v})^{T} \lambda^{\ast} \vec{v} \\   
        \left< \vec{v} \,, A \vec{v} \right> &= \left< \vec{v} \,, \lambda^{\ast} \vec{v} \right> \\  
        \left< \vec{v} \,, \lambda \vec{v} \right> &= \left< \vec{v} \,, \lambda^{\ast} \vec{v} \right>\quad \because A \vec{v} = \lambda \vec{v} \\
        \overline{\lambda} \left< \vec{u} \,, \vec{u} \right> &= \lambda \left< \vec{u} \,, \vec{u} \right> \quad \because \lambda^{\ast} = \overline{\lambda} \mbox{ and } \left< \vec{v} \,, a\vec{u} \right> = \overline{a} \left< \vec{v} \,, \vec{u} \right> \\ 
        \Rightarrow \overline{\lambda} &= \lambda \\
        \Rightarrow \lambda & \in \mathbb{R}
    \end{align*}
\end{proof}
%
\pagebreak
%
\[ \textbf{Group} \]
$\text{Let a, b, c} \in \mathbb{G} $\\
There is binary operation * and satisfies\\

Closure Law\\
$ a*b \in \mathbb{G} $\\

Associative Law\\
$ a*b*c = a*(b*c)$\\

Identity\\
$ \exists \mathit{e} \in \mathbb{G} \text{ such that } \mathit{e}*a = a*\mathit{e} \in \mathbb{G}$\\

Inverse\\
$ \text{If a } \in \mathbb{G}, \exists a^{-1} \in \mathbb{G} \text{ such that } a*a^{-1} = e $\\

XOR over $\mathbb{Z}/\mathbb{Z}_2$ forms a group\\
1. $ a \oplus b = a \oplus b $ commutative\\ 
2. $ a \oplus b \oplus c = a \oplus (b \oplus c) $ associative\\ 
3. $ 0 \oplus a = a $ identity\\ 
4. $ a \oplus a = 0 $ own inverse\\  

\[ \textbf{ Semigroup } \]
Semigroup is a set $S$ and a binary operator $\otimes \colon S \times S \rightarrow S$ that satisfies 
associative property\\ 
\[ \forall \text{ a, b, c} \in S \text{ such as } a \otimes b\otimes c = a \otimes (b \otimes c) \]

\[ \textbf{ Monoid } \]
A monoid is a triple $(S, \otimes, \overline{1})$ \\
1. $\otimes$ is closed associative binary operator on the set $S$ \\
2. $\overline{1}$ is identity element for $\otimes$ \\
$\forall\quad a, b, c \in S$ such as\\
\[ a \otimes b  \otimes c = a \otimes (b \otimes c)   \]
\[ a \otimes \overline{1} = \overline{1} \otimes a =  a  \]

\newpage
\[ \text{Category is an algebraic structure comprises "objects" that are linked by arrows} \] 

\[ \text{ Category } \]
A category is collection of objects and collection of arrow[morphism, map] which have following structure
each arrow has domain and codomain which are object     
\[ f \colon X \rightarrow Y \text{ or } X \xrightarrow{f} Y\]
Composition
\[ f \colon X \rightarrow Y \text{ and } g \colon Y \rightarrow Z \quad g \circ f \colon X \rightarrow Z \quad \]
Composition Identity
\[ f \colon X \rightarrow Y \text{ and } g \colon Y \rightarrow Z  \quad \exists id_y \colon Y \rightarrow Y \mid id_y \circ f = f \text{ and } g \circ id_y = g \]
Associative
\[ f \colon X \rightarrow Y \quad g \colon Y \rightarrow Z \quad h \colon Z \rightarrow W \mid h \circ g \circ f = h \circ (g \circ f) \quad \]

\[ \text{Functor} \]

Functor can be formally defined by a pair of functions $f_1 \mbox{ and } f_0$ 
so that $f_0:Ob(Hask) \rightarrow Ob(Hask)$ and $f_1:Hom(Hask) \rightarrow Hom(Hask)$
where $Hom(Hask)$ refers to the union of all sets $a \rightarrow b$ where $a, b \in Ob(Hask)$
so that the following holds \\

0. $id::a \rightarrow a $ \\
1. If $g::a \rightarrow b $ then $f_1(g)::f_0(a) \rightarrow f_0(b)$ \\
2. For all $a \in \mbox{Ob(Hask)}, \quad f_1(id_a) = id_{f_0(a)} $ \\
3. If $g,h \in \mbox{Hom(Hask)} \mbox{ then } \quad f_1(g \circ h) = f_1(g) \circ f_1(h)$ \\
Given $f_0(a) = \mbox{List } a \quad f_1(g) = \mbox{ map } g   $ \medskip 
Prove $f_0$ and $f_1$ is a Functor \\

\[ \text{ Ring } \]
Let a, b, c $\in  \mathbb{R}$ There are addition and multiplication operations and satisfy associative and distributive laws\\

Associative Law\\
$ a \times b \times c = a \times (b \times c) $\\

Distritutive Law\\
$a \times (b + c) = a \times b + a \times c $\\

Additive inverse\\
For all a in $\mathbb{R}$, there exists -a such that\\
$a + (-a) = 0$\\

Multiplicative identity \\
For all a in $\mathbb{R}$, there exist 1 such that\\
$1a = a$ \\

\[ \text{Division Ring} \] 
Division Ring is a set $F$, together with two operations + and $\times$ 
1. $F$ is abelian group under +
2. The non-zero elements of $F$ form group under $\times$ (not necessary commutative)

\[\text{Group homomorphism(operation preserving)}\]
Given group $(G1, +)$ and $(G2, *)$,for all $a_1, a_2 \in G1$ and $b_1, b_2 \in G2$,\\
if $\phi(a_1 + a_2) = \phi(b_1)*\phi(b_2)$, then $\phi$ is group homomorphism\\

Given $G(\mathbb{R}, +)$ and $(\mathbb{R}, *)$, then $\phi(x) = e^x$ is homomorphism\\ 
Let $a_1, b_1 \in \mathbb{R}$ and $a_2, b_2 \in \mathbb{R}$\\ 
$\phi(a_1+b_1) = e^{a_1 + b_1}$ and $\phi(a_2)*\phi(b_2) = e^{a_2}*e^{b_2} = e^{a_2+b_2}$\\
$\Rightarrow \phi(a_1 + b_1) = \phi(a_2)*\phi(b_2)$\\
$\Rightarrow \phi(x) = e^{x}$ is homomorphism for $G(\mathbb{R}, +)$ and $G(\mathbb{R}, *)$

\[\text{Normal Group} \]
if $N$ is subgroup of $G$, and if $gH = Hg \quad \forall g \in G$, then $H$ is normal

\[ \text{Coset} \]
if $N$ is subgroup of $G$, and if $gH = \{gh: \forall g \in G \}$, then $gH$ is left coset of $H$ in $G$ with repsect to $g$.\\
Similarly, if $Hg = \{hg: \forall g \in G \}$, then $Hg$ is right coset of $H$ in $G$ with repsect to $g$.\\
\begin{center}
\includegraphics[width=10cm,scale=1]{/Users/cat/myfile/github/image/coset.jpg}\\
\end{center}

\newpage
Ring homomorphism(operation preserving)
Let $\phi$ is a function between two rings $R$, then $\phi$ is a $\mathit{ring}$ homomorphism if
for all $a \in R$ and $b \in R$ 
\[\phi(a+b) = \phi(a) + \phi(b)\] 
\[\phi(ab) = \phi(a)\phi(b)\]
and \[\phi(1) = 1\]
e.g. $G(\mathbb{N}, +)$ and $G(\mathbb{Z}/\mathbb{Z}_5, +)$\\

Let $\phi: \mathbb{C} \rightarrow \mathbb{C}$ be the map send a complex number to its complex conjugate. Then $\phi$ is an automorphism of $\mathbb{C}$. 
$\phi$ is its own inverse.\\

\begin{equation}
\begin{aligned}
\phi(z) &= \overline{z}\\
\phi(z_1 + z_2) &= \overline{z_1 + z_2}\\
\overline{z_1 + z_2} &= \overline{z_1} + \overline{z_2}\\
\phi(z_1 z_2) &= \overline{z_1 z_2}\\
\overline{z_1 z_2} &= \overline{z_1} \cdot \overline{z_2} \nonumber\\
\phi(\phi(z)) &= z\\
\end{aligned}
\end{equation}

Let $\phi: \polyringr{x} \rightarrow  \polyringr{x}$ be the map that send $f(x)$ to $f(x+1)$. Then $\phi$ is an automorphism of $\polyringr{x}$.   
The inverse map sends $f(x)$ to $f(x-1)$\\

\[ \text{Ideal}\\ \]
Let $R$ be a ring and let $I$ is additive subgroup of $R$, then $I$ is called an ideal of $R$ and write $I \triangleleft R$ 
\quad if $\forall a \in I$ and $\forall r \in R $, and $ ar \in I$ and $ra \in I$\\

Example\\
$R = (\mathbb{N}, +)$ and $I = (2k, +) \quad k \in \mathbb{N}$\\

Let $I$ be a kernal of $\phi$, then $I$ is an ideal of $R$\\
Let $a \in I$ and $r \in R$, then $\phi(ra) = \phi(r)\phi(a)$\\
$I$ is kernal of $\phi$ $\Rightarrow \phi(a) = 0 \therefore \phi(ra) = 0, \therefore ra \in I$\\

$\textbf{If }  \gcd(a, b) = 1 \textbf{ and }  a \vert bc \quad$  $\Rightarrow a \vert c$ \\
$\textbf{Proof}$ \\
$\gcd(a, b) = 1  $\\
$\Rightarrow ma+nb = 1\quad m, n \in \mathbb{N} $ \\
$\Rightarrow mac + nbc = c$ \\
$a \vert bc \Rightarrow ak = bc \quad k \in \mathbb{N} $ \\
$\Rightarrow mac + n(ak)=c \quad    (ak=bc) $ \\
$\Rightarrow a(mc + nk) = c$  \\
$\Rightarrow a \vert c $ \\
\\
$\textbf{If } \gcd(a, b) = 1 \Rightarrow ma + nb = 1 \quad m, n \in \mathbb{N}$\\
$\textbf{Proof}$\\
\\
$\textbf{Prove there is infinite prime}$\\
\\
$\textbf{Prove all the eigeivalues}\quad  \lambda \geq  0  \textbf{ if the matrix is symmetic}$\\
\\
$\textbf{If the determine of matrix } \det{A} > 0 \iff \textbf{the matrix is invertable}$\\
\\
Efficient Algorithm to compute Fibonacci Number \\
Fibonacci Sequence   
$F_{n} = F_{n} + F_{n-1}$ with $F_{0} = 0$
$F_{1} = 1$ \\
(1) Navier algorithm with recursion in $O(2^n)$ \\
(2) Use Dynamic Algorithm in $O(n)$\\
(3) Use matrix with repeated squaring to compute Fibonacci Sequence in $O(\log{n})$
\[\left(\begin{array}{cc} F_{n+1} & F_{n} \\ F_{n} & F_{n-1} \end{array} \right)^n =  \left(\begin{array}{cc}1 & 1 \\ 1 & 0 \end{array} \right)^n \]
\\
\newpage

\begin{flushleft}
$\text{Visual proof}$\\
$ (\sum_{k=1}^{n} k)^{2} = \sum_{k=1}^{n} k^{3}$\\
\includegraphics[width=10cm,scale=1]{/Users/cat/myfile/github/image/cubeproof.png}
\end{flushleft}

\begin{flushleft}
$\text{Show the sum of odd number are square number}$
\medskip
1\\
1 + 3\\
1 + 3 + 5 \\
1 + 3 + 5 + ... + (2k+1) \\
\medskip
$S = \sum_{k=1}^{n} (2k-1)$ \\
$S = \sum_{k=1}^{n} 2k - \sum_{k=1}^{n} 1$ \\
$S = 2(\sum_{k=1}^{n} k) - n $ \\
$S = 2 \frac{(1+n)n}{2} - n$ \\
$S = (1+n)n - n$ \\
$S = n^2 $ \\
\end{flushleft}

composition function \\
$g \circ f \circ h $ \\
$g \circ f \colon A\to B$ \\

\pagebreak
\mbox{Find the sum of sequence of squares}
    \[ s = \sum_{k=1}^{n} k^2 \]
\begin{equation}
\begin{aligned}
    & \sum_{k=1}^{n} ((k+1)^3-k^3) = \sum_{k=1}^{n} (k+1)^3 - \sum_{k=1}^{n} k^3  \\
    & \Rightarrow 2^3 + 3^3 + ... + n^3 + (n+1)^3 - (1 + 2^3 + 3^3 + ... + n^3) = (n+1)^3 -1  \\
    & \Rightarrow  \sum_{k=1}^{n}(k+1)^3 - \sum_{k=1}^{n}k^3 =(n+1)^3 - 1  \\
    & \Rightarrow (n+1)^3-1 = \sum_{k=1}^{n}(3k^2+3k+1)   \\
    & \Rightarrow (n+1)^3-1 = 3\sum_{k=1}^{n} k^2 +  3\sum_{k=1}^{n} k + n   \\
    & \Rightarrow (n+1)^3-1 = 3\sum_{k=}^{n} k^2 + (n+1) n \frac{3}{2} + n  \\
    & \Rightarrow (n+1)^3-1 -  (n+1) n \frac{3}{2} - n= 3\sum_{k=1}^{n} k^2   \\
    & \Rightarrow (n+1)((n+1)^2-n \frac{3}{2})-(n+1) = 3\sum_{k=1}^{n} k^2  \\
    & \Rightarrow (n+1)( (n+1)^2 -n\frac{3}{2}-1) = 3\sum_{k=1}^{n} k^2  	 \\
    & \Rightarrow (n+1)( n^2+1+2n-n\frac{3}{2} - 1) = 3\sum_{k=1}^{n} k^2  	 \\
    & \Rightarrow (n+1)(n^2 + \frac{1}{2}n) = \sum_{k=1}^{n} k^2  \\
    & \Rightarrow \frac{1}{2}(n+1)(2n^2+n)=\sum_{k=1}^{n} k^2  \\
    & \Rightarrow \frac{1}{6}n(n+1)(2n+1) = \sum_{k=1}^{n} k^2  \nonumber \\
\end{aligned}
\end{equation}

\pagebreak
\begin{align*}
    & \mbox{Vector Space} \\
    & \noindent \text{Let }\vec{u}, \vec{v}, \vec{w} \in \vec{V} \text{ and scalars } \alpha, \beta \in \mathbb{F} \\
    & \mbox{Closure} \\ 
    & \vec{u} + \vec{v} \text{ and } \in \vec{V} \\
    & \mbox{Associative Law} \\
    & \vec{u} + \vec{v} + \vec{w} = \vec{u} + (\vec{v} + \vec{w}) \\
    & \mbox{Commutative Law} \\
    & \vec{u} + \vec{v} = \vec{v} + \vec{u}  \\
    & \mbox{Identity element of addition}\\
    & \vec{0} \in \vec{V} \text{ such that } \vec{0} + \vec{u} = \vec{u}\\
    & \mbox{Inverse element of addition}\\
    & \exists -\vec{u} \text{ such that } \vec{u} + (-\vec{u}) = \vec{0}\\
    & \mbox{Identity element of scalar multiplication}\\
    & \exists \mathit{1} \in \mathbb{F} \text{ such that } \mathit{1}\vec{u} = \vec{u}\\
    & \mbox{Distributivity of scale multiplication with respect to vector addition}\\
    & \alpha(\vec{u} + \vec{v}) = \alpha\vec{u} + \alpha\vec{v}\\
    & \mbox{Distributivity of scale multiplication with respect to field addition}\\
    & (\alpha + \beta)\vec{u} = \alpha\vec{u} + \beta\vec{u}\\
\end{align*}

\pagebreak
\begin{align*}
    & \mbox{Linear Transformations} \\
    & \mbox{A function } \mathit{T}: \mathbb{R}^n \rightarrow \mathbb{R}^m \mbox{ is called linear transformation, if it satisfies} \\
    & \mathit{T} ( \vec{u} + \vec{v} ) = \mathit{T}(\vec{u}) + \mathit{T}(\vec{v}) \quad \forall \; \vec{u} \,, \vec{v} \in \mathbb{R}^n\\
    & \mathit{T} ( \lambda \vec{u} ) = \lambda \mathit{T}(\vec{u}) \quad \mbox{all scalars } \lambda \\
\end{align*}

\pagebreak
Definition of Affine Space\\
An affine space is a set of points that admits free transitive action of a vector space $\vec{V}$ That is, there is a map $X \times \vec{V} \rightarrow X:(x, \vec{v}) \mapsto x + \vec{v}$, called translation by a vector $\vec{v}$, such that\\
1. Addition of vectors corresponds to composition of translation, i.e., for all $x \in X \text{ and } \vec{u}, \vec{v} \in \vec{V}, (x + \vec{u}) + \vec{v} = x + (\vec{u} + \vec{v})$\\ 
2. The zero vector $\vec{0}$ acts as the identity vector, i.e., for all $x \in X, x + \vec{0} = x$\\
3. The action is transitive, i.e., for all $x, y \in X, \text{ exists } \vec{v} \in \vec{V} \text{ such that } y = x + \vec{v}$\\
4. The dimension of X is the dimension of vector space translations, $\vec{V}$\\\\
Or There is unique map\\
$X \times X \rightarrow \vec{V}:(x, y) \mapsto y - x \text{ such that } y = x + (y - x) \text{ for all }x, y \in X$
It furthermore satifies\\ 
1. For all $x, y, z \in X$, z - x = (z - y) + (y - x)\\
2. For all $x, y, \in X$ and $\vec{u}, \vec{v} \in \vec{V}$, $ (y + \vec{v}) - (x + \vec{u}) = (y - x) + (\vec{v} - \vec{u})$\\
3. For all $x \in X, x - x = \vec{0}$\\
4. For all $x, y \in X, y - x = -(x - y)$\\

Affine Space from linear system equation\\
Consider an $(m \times n)$ linear sytem equations\\\\
$\sum_{k=1}^{n} a_{i k} x_{k} = c_{i}, (1 \leq i \leq m) \quad\quad\quad \text{(1)}$\\\\
where $d = n - rank(M), c_{i} \ne \vec{0} \in \mathbb{R}^{m}$\\
When the system has at least one solution $x_{p}$ then the full set of solution is a d-dimension affine space
$A \subset \mathbb{R}^{n}$\\ 
Since $x_{p} \in A, \text{ we can declare point } x_{p} \text{ as origin of } A$ and then introduct A coordinates as follows:homogenous system\\

$\sum_{k=1}^{n} a_{i k} x_{k} = \vec{0} (1 \leq i \leq m)$\\\\
$\Rightarrow dim(Ker(M)) = d \quad \text{(Rank Theorem)}$\\
$\text{(1) has d-linear independent solution } \vec{b_{j}} \in \mathbb{R}^{n} \quad\quad (1 \leq j \leq d)$\\
$\text{Affine Space }\mathit{A}$ can be written as\\\\ 
$A = \Big\{ x_{p} + \sum_{j=1}^{d}\alpha_{j}\vec{b_{j}} \quad \mid \quad \alpha_{j} \in \mathbb{R} \quad\quad (1 \leq j \leq d)\Big\} $\\\\
$\text{The } \alpha_{j} \text{ can be served as coordinates in A, so that A looks as it were a d-dimension coordiate space.}$\\ 
$\text{But note that addition(+) in the space refers to the chosen point } x_{p}, \text{ and not to the origin of the base vector space}$\\

$
        \begin{bmatrix}
        1 & 2 & 3 \\
        4 & 5 & 6    
        \end{bmatrix}
        \left[
        \begin{array}{c}
        x_1 \\
        x_2 \\
        x_3 
        \end{array}
        \right] = 
        \left[ 
        \begin{array}{c}
        1 \\ 
        2 
        \end{array}
        \right]
$
\newpage

\noindent
Theorem 1\\
The image of transformation is spanned by the image of the any basis of its domain. For $T:\vec{V} \rightarrow \vec{W}, \text{ if } \beta=\{ \vec{b_1},\vec{b_2},...,\vec{b_n} \} \text{ is a basis of }\vec{V}, 
\text{ then }T(\beta) = \{ T(\vec{b_1}), T(\vec{b_2}), ... ,T(\vec{b_n})\} \text{ spans the image of }T$\\

\noindent
Proof\\
For all $\vec{v} \in \vec{V}, \vec{v} = \alpha_1\vec{b_1} + \alpha_2\vec{b_2} + ... + \alpha_n\vec{b_n}$\\
$\Rightarrow T(\vec{v}) = T(\alpha_1\vec{b_1} + \alpha_2\vec{b_2} + ... + \alpha_n\vec{b_n})$\\
$\Rightarrow T(\vec{v}) = \alpha_1 T(\vec{b_1}) + \alpha_2 T(\vec{b_2}) + ... + \alpha_n T(\vec{b_n})$\\
$\Rightarrow \{ T(\vec{b_1}), T(\vec{b_2}),...,T(\vec{b_n})\} \text{ spans the image of }T$\\

\noindent
Rank Theorem\\
If the domain is finite dimension, then the dimension of domain is the sum of rank and nullity of the transformation\\
$\text{Let } T:\vec{V} \rightarrow \vec{W} \text{ be a linear transformation },\text{let n be the dimension of }\vec{V},$\\
$\text{let k be nullity of }T \text{ and let k be the rank of }T$\\
$\text{Show } n = k + r$\\

$\text{Let }\beta = \{ \vec{b_1}, \vec{b_2},...,\vec{b_k}\} \text{ be the basis of kernal of }T, \text{ the basis can be extended to } \gamma = \{ \vec{b_1}, \vec{b_2},...,\vec{b_k}, \vec{b_{k+1}},...,\vec{b_n}\}$\\
$\text{let }\vec{v} \in \vec{V} \Rightarrow \vec{v} = \alpha_1 \vec{b_1} + \alpha_2 + \vec{b_2} +,..., + \alpha_k \vec{b_k} + \alpha_{k+1}\vec{b}_{k+1}+,...,+\alpha_{n}\vec{b_n}$\\
$\text{Let }T(\vec{v}) = T(\alpha_1 \vec{b_1} + \alpha_2 + \vec{b_2} +,..., + \alpha_k \vec{b_k} + \alpha_{k+1}\vec{b}_{k+1}+,...,+\alpha_{n}\vec{b_n}) = \vec{0}$\\
$\Rightarrow \vec{v} = \alpha_1 \vec{b_1} + \alpha_2 + \vec{b_2} +,..., + \alpha_k \vec{b_k} + \alpha_{k+1}\vec{b}_{k+1}+,...,+\alpha_{n}\vec{b_n} \in \ker(T) \quad\quad \text{(1)}$\\
$\because \vec{v} = \sigma_1 \vec{b_1} + \sigma_2 + \vec{b_2} +,..., + \sigma_k \vec{b_k} \in \ker(T) \quad\quad \text{(2)}$\\
$(1) - (2) \Rightarrow \vec{0} = (\alpha_1-\sigma_1)\vec{b_1} + (\alpha_2 - \sigma_2)\vec{b_2}+,...,+ (\alpha_k - \sigma_k)\vec{b_k}+   \alpha_{k+1}\vec{b}_{k+1}+,...,+\alpha_{n}\vec{b_n} $\\
$\because \vec{b}_{1}, \vec{b}_{2},...,\vec{b}_{k},\vec{b}_{k+1}, \vec{b}_{k+2},...,\vec{b_n} \text{ are linearly independent}$\\
$\therefore \alpha_{k+1}, \alpha_{k+2}, ... , \alpha_{n} \text{ are all zero} \quad\quad \text{(3)}$\\
$T(\vec{v}) = T(\alpha_1 \vec{b_1}) + T(\alpha_2 \vec{b_2}) +,..., + T(\alpha_k \vec{b_k}) + T(\alpha_{k+1}\vec{b}_{k+1})+,...,+T(\alpha_{n}\vec{b_n}) = \vec{0}$\\
$T(\vec{v}) = \alpha_1 T(\vec{b_1}) + \alpha_2 T(\vec{b_2}) +,..., + \alpha_k T(\vec{b_k}) + \alpha_{k+1}T(\vec{b}_{k+1})+,...,+\alpha_{n}T(\vec{b_n}) = \vec{0}$\\
$\because \beta = \{ \vec{b_1}, \vec{b_2},...,\vec{b_k}\} \text{ is the basis of kernal of }T$\\
$\therefore T(\vec{b_1}) = \vec{0},..., T(\vec{b_k}) = \vec{0}$\\
$\therefore T(\vec{v}) = \alpha_{k+1}T(\vec{b}_{k+1})+,...,+\alpha_{n}T(\vec{b_n}) = \vec{0} \quad\quad \text{(4)}$\\
$\text{(3) and (4)} \Rightarrow \{ T(\vec{b}_{k+1}), T(\vec{b_{k+2}}), ... , T(\vec{b_{n}}) \} \text{ are linearly independent}$\\
$\Rightarrow \dim(\vec{V}) = \text{ nullity(T) } + \text{ rank(T) } \text{ or }$\\
$\Rightarrow \dim(\vec{V}) = \dim(\ker(T)) + \dim(\text{img(T)}) $\\
$\Rightarrow n = k + r \quad \square$\\

\noindent
Affine plane\\
Affine plane is a set, whose elements are called points, and a set of subset, called lines, satifying the following three axioms:\\
1. Given two distinct points P and Q, there is one and only one containing both P and Q.\\
2. Given a line l, and a point P not in l, there is one and only one line m which is parallel to l and which passes through P.\\
3. There exists three non-collinear points.\\ 

\noindent
A coordinate system in an affine space $(\mathbf{X}, \mathbf{V}) \text{ consists of a point } \mathbf{O} \in \mathbf{X}$ is called origin, and basis 
$\vec{v_1},...,\vec{v_n}\quad\text{ for } \vec{V} \text{ Any point }\mathbf{x} \in \mathbf{X} \text{ can be written as}$\\

$\mathbf{x} - \mathbf{O} = \mathbf{x} - \mathbf{O}$\\
$\Rightarrow \mathbf{x} = \mathbf{O} + (\mathbf{x} - \mathbf{O}) = \mathbf{O} + \sum_{k=1}^{n}\mathit{x_k}\vec{v_k}$\\
$\text{where the numbers }\mathit{x_1},...,\mathit{x_n}\text{ are the coordinates for vector }\mathbf{x} - \mathbf{O}\text{ with respect to the basis }\vec{v_1},...\vec{v_n}.$\\
$\text{They are now also called the coordinates for }\mathbf{x} \text{ with respect to the coordinate system }\mathbf{O}, \vec{v_1},...,\vec{v_n}$\\

\noindent
Projective plane\\
A projective plane $\mathbb{S}$ is a set, whose elements are called points, and a set of subset, called lines, satifying the following four axioms.\\
1. Two distinct points meets P, Q of S lie on one and only one line.\\ 
2. Any two lines meet in at least one point.\\
3. There exist three non-colinear points\\
4. Every line contains at least three points.\\

\noindent
Manifold\\
$\text{An subset }\mathbf{S} \text{ of } \mathbb{R}^{m} \text{ is called a manifold of dimension of d if every point p of }\mathbf{S} \text{ has a neighbourhood in }$
$\mathbf{S}\text{ which is homeomorphic to an open set of }\mathbb{R}^{d}$\\

\noindent
Homeomorphrism\\
$\text{Let S be a subset of }\mathbb{R}^{m} \text{ and sp be the subset of }\mathbb{R}^{n}. \text{ A map } \mathit{f}: \mathbb{R}^{m} \mapsto \mathbb{R}^{n}$\\
$\text{ is called homeomorphism if }\mathit{f} \text{ is continuous and bijective and }\mathit{f}^{-1} \text{ is continuous}$\\


\noindent
Definition Open and Close Sets\\
1. $\mathbf{S}$ is said to be open set if every point of $\mathbf{S}$ is an interior of $\mathbf{S}$\\
2. $\mathbf{S}$ is said to be closed set if $\mathbb{R}\setminus\mathbf{S}$ is open\\  

\noindent
Proposition\\
1. $\mathbf{S} \text{ is open if there exists } \delta > 0 \text{ such that } (\mathit{s} - \delta, \mathit{s} + \delta) \subseteq \mathbf{S}$\\
2. $\mathbf{S} \text{ is open if any }\mathit{s} \in \mathbf{S} \text{ there exists a neighbourhood of }\mathit{s} \text{ included in }\mathbf{S}$\\  


\noindent
Terminology\\
$\mathcal{M} \text{ Set (ZFC) book}$\\
$\mathcal{Q} \text{ topology =: set of open set}$\\
$(\mathcal{M}, \mathcal{Q}) \text{ toplogy space}$\\
$\mathcal{U} \in \mathcal{Q} \iff \text{ all } \mathcal{U} \subseteq \mathcal{M} \text{ and open set}$\\
$\mathcal{M} \setminus \mathcal{A} \iff   \text{ all }    \mathcal{A} \subseteq \mathcal{M} \text{ closed set}$\\
$\text{open } \centernot\implies \text{ closed}$\\
$\text{open } \centernot\impliedby \text{ closed}$\\

\noindent
Definition of Inner Product
\noindent
Positivity\\
$\langle\vec{v}, \vec{v}\rangle \geq 0$\\
$\langle \vec{v} , \vec{v} \rangle = \vec{0} \iff \vec{v} = \vec{0}$\\

\noindent
Bilinearity\\
$\langle c_{1}\vec{v_1} + c_{2}\vec{v_2}, \vec{v_3}\rangle = c_{1}\langle \vec{v_1}, \vec{v_3}\rangle + c_{2}\langle\vec{v_2}, \vec{v_3} \rangle$\\

\noindent
Conjugate Symmetic\\
$\langle \vec{v_1}, \vec{v_2} \rangle = \overline{\langle \vec{v_2}, \vec{v_1} \rangle}$


\noindent
Proof  Cauchy-Schwarz Inequality by picture\\
$\vert a \cdot b \vert \leq \lVert a \lVert \lVert b \lVert$


\pagebreak
Calculate the Excel Sheet Row number algorithm
Latex Environment has different mode\\
{\it Math mode}\\
{\it Text mode}\\
{\it Command mode}\\


Elliptic Curve and Group Structure Conic Curve\\
$
    \begin{bmatrix}
    x & y & z 
    \end{bmatrix}
    \begin{bmatrix}
    a & b & c \\
    d & e & f \\   
    g & h & i \\   
    \end{bmatrix}
    \left[
    \begin{array}{c}
    x \\
    y \\
    z 
    \end{array}
    \right] = 
    \left[ 
    \begin{array}{c}
    0  
    \end{array}
    \right]
$\\
$ ax^{2} + ey^{2} + iz^{2} + (b+d)xy + (c+g)xz + (f+h)yz = 0$\\
\\
$\mathit{(x, y, z)} = \mathit{(x/z, y/z, 1)} (z \neq 0)$\\ 
\\

\noindent
$
\begin{array}{lcl}
ax^{2} + ey^{2} + (b+d)xy + (c+g)x + (f+h)y + i &=& 0\\
ax + by + c &=& 0
\end{array}
$\\
sub (1) into (2) 
$\Rightarrow ax^{2} + bx + c = 0$\\
$\Rightarrow x = \frac{-b \pm \sqrt{b^{2} - 4ac}}{2a}$
\\

Exponential backoff algorithm\\
\\
$\frac{1}{N+1} \sum_{i=1}^{N}i$\\
For example, the expected backoff time for the third collision, one could 
calculate the maximum backoff time, N\\
$N = 2^{c} - 1 (c = 3)$
$N = 7$\\
Calculate the mean of backoff time for the third collision(c=3)\\
$\mathbf{E(c)} = \frac{1}{N+1}\sum_{i=0}^{N} i$\\
$\mathbf{E(c)} = \frac{1}{N+1}\sum_{i=0}^{N} \Rightarrow \frac{1}{N+1} \frac{N(N+1)}{2} = \frac{N}{2}$\\
$\mathbf{E(3)} = \frac{1}{7+1}\sum_{i=0}^{N} i = \frac{1}{8}(0+1+2+3+4+5+6+7) = \frac{28}{8}$\\
$\mathbf{E(3)} = 3.5$\\

\begin{proof}
If x is rational and y is irrational, then x + y is irrational \\
Assume $x + y$ is rational \\
    $\Rightarrow \frac{n}{m} = x + y  \quad m, n \in  \mathbb{N}$ \\ 
    $\Rightarrow \frac{n}{m} - \frac{n_1}{m_1} = y \quad m_1, n_1 \in  \mathbb{N}$ \\
    $\Rightarrow \frac{n m_1}{m m_1} - \frac{n_1 m}{m m_1} = y$  \\
    $\Rightarrow \frac{n m_1 - n_1 m}{m m_1} = y$  \\
    $\Rightarrow \mbox{y is rational}$  \\
    $\Rightarrow \mbox{this contracts y is irrational}$ \\
    $\Rightarrow \mbox{y is irrational}$ \\
\end{proof}

Prove Square root of two is irrational
$\sqrt{2} \notin \mathbb{Q}$\\
$\text{Assume } \sqrt{2} \in \mathbb{Q}$\\
$\text{let }n = min\{ n \in \mathbb{N} \mid n*\sqrt{2} \in \mathbb{N}\}$\\
$\Rightarrow n*(\sqrt{2} - 1)*\sqrt{2} \in \mathbb{Q}$\\
$\because \sqrt{2} - 1 < 1$\\
$\Rightarrow n*(\sqrt{2} - 1)*\sqrt{2} < n*\sqrt{2}$\\
$\Rightarrow n*(\sqrt{2} - 1) < n \text{ such as } n*(\sqrt{2} - 1)*\sqrt{2} \in \mathbb{N}$\\
$\Rightarrow \text{This contracts our assumption } \quad \square$\\

\pagebreak
\noindent
\[ \text{Prove Square root of 2 is irrational} \text{ [ Geometric proof ] }\]
$\text{Assume } \sqrt{2} \in \mathbb{Q}$\\
$\Rightarrow \frac{a}{b} = \sqrt{2} \quad a, b \in \mathbb{N} \text{ and } \gcd(a, b) = 1, a > b$\\

given a right issoseles triangle\\
$AB = AC = AE, \quad FE \text{  tangles to arc at point } E $\\
$\Rightarrow AF = EF$\\
$\text{Let } AB = AC = 1$\\
$\Rightarrow BC = \sqrt{2}$\\
$\Rightarrow FB = 1-EB = 1-(\sqrt{2} - 1) = 2-\sqrt{2}$\\
$\because \frac{AC}{CB} = \frac{EB}{FB} = \frac{1}{\sqrt{2}} = \frac{\sqrt{2}-1}{2-\sqrt{2}}$\\
$\therefore \frac{1}{\sqrt{2}} =\frac{1}{\frac{a}{b}} = \frac{\sqrt{2}-1}{2-\sqrt{2}} = \frac{\frac{a}{b} - 1}{2-\frac{a}{b}}$\\
$\therefore \frac{b}{a} = \frac{ \frac{a-b}{b} } {\frac{2b - a}{b}} = \frac{a-b}{2b-a}$\\
$\because \sqrt{2} < \sqrt{4} = 2 \therefore  a < 2b $\\
$\Rightarrow  \color{red} a-b < b $\\
$\because 2a > 2b$\\
$\Rightarrow  \color{red}2b -a < a$\\
$\text{That contracts our assumption } \gcd(a, b) = 1$\\
$\Rightarrow \frac{a}{b} \notin \mathbb{Q}$

\pagebreak
\noindent
Geometric proof: square root of two is irrational\\
$\text{Given an isosceles right triangle from above and let } \gcd(a, b)=1, \text{ from Pythagorean theorem}$\\
$\Rightarrow a^2 = b^2 + b^2$\\
$\Rightarrow \sqrt{2} = \frac{a}{b}$\\

singular point on affine plane curve\\
$\text{If } p \in (x, y) \text{ and } \frac{df}{dx}, \frac{df}{dy} \text{ are undefined on } p \in (x, y), \text{ then } p\in (x, y) \text{ is singular point}$\\


\[\text{Eisenstein series}\]
$L = [w_1, w_2] \in \mathbb{C}, G(L) = \sum_{w\in L\setminus \{0, 0\}} \frac{1}{w^k}$\\
$\text{Let lattices } L=[1, \tau] \text{ and parametrized by }\tau  \text{ in the upper half plane } \mathbb{H} = \{ z \in\mathbb{C} : \Im(z) > 0 \} $\\
$G(L) = G([1, \tau]) =G(\tau) = \sum_{m, n \in \mathbb{Z}}' \frac{1}{(m+n\tau)^k}$\\

\noindent
\[\text{Show } G_{k}(\tau + 1) = G_{k}(\tau)\]
$G_{k}(\tau + 1) = \sum_{m, n \in \mathbb{Z}}' \frac{1}{(m+n(\tau+1))^k} = \sum_{m, n \in \mathbb{Z}}' \frac{1}{(m + n + n\tau)^k} = \sum_{m', n \in \mathbb{Z} }' \frac{1}{(m' + n\tau)^k}$\\
$\Rightarrow G_{k}(\tau + 1) = G_{k}(\tau)$\\

\noindent
\[\text{Show } G_{k}(\tau) = \tau^{-k} G_{k}(\frac{-1}{\tau})\]
$\tau^{-k} G_{k}(\frac{-1}{\tau}) =  \tau^{-k} \sum_{(m, n \in \mathbb{Z} )}' \frac{1}{(m+\frac{n}{\tau})^{k}}$\\
$\tau^{-k} \sum_{m, n \in \mathbb{Z}}' \frac{1}{(m + \frac{n}{\tau})^{k}} = \tau^{-k} \sum_{m, n \in \mathbb{Z}}' \frac{1}{(\frac{m\tau}{\tau} + \frac{n}{\tau})^{k}} = \sum_{m, n \in \mathbb{Z}}' \textcolor{red}{(\tau^{-k}\tau^{k})} (m\tau + n)^{-k})$\\
$\Rightarrow G_{k}(\tau) = \tau^{-k} G_{k}(\frac{-1}{\tau})$\\


\noindent
\[\text{Show }G_{k}(\tau) = 0 \text{ if } k = (2j+1) \quad j \in \mathbb{Z} \]
$\text{For each }\omega = (m+n\tau) \in L, \text{there exists } -\omega = -(m+n\tau) \in L $\\
$\because \omega ^{-(2J+1)} + (-1)^{-(2J+1)}\omega^{-(2j+1)} = 0$\\
$\therefore \sum_{m,n \in \mathbb{Z}}' (m+n\tau)^{-k} = 0\quad \text{ for all } k \in (2j+1)$\\

\begin{theorem}
Show for any lattices $L$, the sum $\sum_{\omega \in \mathit{L}}' \frac{1}{\omega^{k}} \text{ converges absoluately for all } k > 2$
\end{theorem}


\pagebreak
\noindent
\[\text{Show }\frac{1}{(1-x)^2} = \sum_{n=0}^{\infty} (n+1)x^n \quad |x| < 1\]
Proof:
$\text{For all } |x| < 1, \text{ power series expansion} $\\
\[(\frac{1}{1-x}\big) = (\sum_{n=0}^{\infty} x^n)\]\\
Differentiate both sides\\
\begin{equation}
\begin{aligned}
(\frac{1}{1-x}\big)' &= (\sum_{n=0}^{\infty} x^n)' \\
(\frac{1}{1-x}\big)' &= (1-x)^{-2}\\ 
\sum_{n=1}^{\infty} nx^{n-1} &= \sum_{i=0}^{\infty} (i+1)x^{i} \quad (n-1=i)\\
\frac{1}{(1-x)^2} &= \sum_{n=0}^{\infty} (n+1)x^{n} \quad \text{ sub }(i = n)\\
\end{aligned}
\end{equation}

\noindent
\[\text{Weierstrass function } \wp \text{-function of lattice L is defined by} \]
$\wp(z) = \wp(z; L) = \frac{1}{z^2} + \sum_{w \in L}' \big[ \frac{1}{(z-\omega)^2} + \frac{1}{\omega^2} \big]$\\

\noindent
Holomorphic\\
A function $f(z)$ defined on some open neibourhood of a point $z_{0} \in \mathbb{C}$ is said to be holomorphic at $z_{0}$ if the complex derivative 
\[ f'(z_{0}) = \lim_{h \to 0 } \frac{f(z_{0} + h) - f(z_{0})}{h} \] exists.
We said $f$ is holomorphic on an open set $\Omega$ if it is holomorphic at every $z_{0} \in \Omega$ and we said $f$ is holomorphic in a closed set $\mathbf{C}$ if it is holomorphic on some open set $\Omega$ containing $\mathbf{C}$.
Functions are holomorphic on all of $\mathbb{C}$ are said to be $\mathit{entire}$\\

\noindent
\[ \text{Show } f(z) = z^2 \text{ is holomorphic} \]
\begin{equation}
\begin{aligned}
f'(z) &= \lim_{h \to 0} \frac{(z+h)^2 - z^2}{h} \\
f'(z) &= \lim_{h \to 0} \frac{z^2 + 2hz + h^2 - z^2}{h}\\
f'(z) &= \lim_{h \to 0} \frac{2hz + h^2}{h} \nonumber \\
f'(z) &= \lim_{h \to 0} 2z + h\\
f'(z) &= 2z
\end{aligned}
\end{equation}

\pagebreak
\noindent
Differential\\
A function $f(x)$ is differential on $x_{0} \in \mathbb{R}$ if \\
\[  f'(x_{0}) =  \lim_{h \to 0} \frac{f(x_{0} +h) - f(x_{0})}{h} \]
exists\\

\[\text{Show } f(x) = x^{2} \text{ is differentiable for all } x  \in \mathbb{R} \]
\begin{equation}
\begin{aligned}
f'(x) &=  \lim_{h \to 0} \frac{f(x +h) - f(x)}{h}\\
f'(x) &= \lim_{h \to 0} \frac{x^2 + h^2 + 2xh - x^2  }{h} \nonumber \\
f'(x) &= \lim_{h \to 0} \frac{h^2 + 2xh}{h}\\
f'(x) &= \lim_{h \to 0} h+2x\\     
f'(x) &= 2x \quad \text{ for all } x \in \mathbb{R}
\end{aligned}
\end{equation}

\pagebreak
Elliptic Curve \\
\begin{equation}
\begin{aligned}
S(n) &= \sum_{k=1}^{n} k^2 \\
S(n) &= \frac{(2n+1)(n+1)n}{2 \times 3} \\
\mbox{Let } y^2 = S(n) \mbox{ and } x=n \nonumber \\
  y^2 &= \frac{1}{6}x(x+1)(2x+1) 
\end{aligned}
\end{equation} 

\begin{center}
\begin{asy}
    unitsize(1cm);
    draw((-4,0) -- (4,0), arrow=Arrow(TeXHead));
    draw((0,-4) -- (0,4), arrow = Arrow(TeXHead));

    unitsize(1cm);
    import contour;
    real f(real x, real y) { return 6*y^2 -
        x*(x+1)*(2*x+1); }
    guide[][] thegraphs = contour(f,
        a=(-4,-4), b=(4,4), new real[] {0});

    /* The next line draws the first (and
        only) entry in thegraphs. This entry
        is itself an array, since it
        represents a disconnected path. */

    draw(thegraphs[0]);
\end{asy} 
\end{center}

$ (x, y)=(0, 0), (-1, 0),(-\frac{1}{2}, 0)$ are on the curve\\

\pagebreak
\begin{proof}
How to derive  $y^2 = x'^3 + Ax' + B$ from $y^2 = x^3 + bx^2 + cx + d$ \\
We use following trick similar to \textbf{Completing the square} in quadratic polynomial\\
\noindent
\begin{equation}
\begin{aligned}
 (x+ \frac{b}{3})^3 &=  x^3 + 3x^2 \frac{b}{3} + 3x (\frac{b}{3})^{2} + (\frac{b}{3})^{3} \\
 (x+ \frac{b}{3})^3 &=  x^3 + bx^2 + 3x (\frac{b}{3})^{2} + (\frac{b}{3})^{3}  \\
 (x+ \frac{b}{3})^3 - [3x (\frac{b}{3})^{2} + (\frac{b}{3})^{3}] &=  x^3 + bx^2 + [3x (\frac{b}{3})^{2} + (\frac{b}{3})^{3}] - [3x (\frac{b}{3})^{2} + (\frac{b}{3})^{3}] \\
 (x+ \frac{b}{3})^3 &=  x^3 + 3x^2 \frac{b}{3} + 3x (\frac{b}{3})^{2} + (\frac{b}{3})^{3} \\ 
 (x+ \frac{b}{3})^3 &=  x^3 + bx^2 + 3x (\frac{b}{3})^{2} + (\frac{b}{3})^{3} \\ 
 (x+ \frac{b}{3})^3 - [3x (\frac{b}{3})^{2} + (\frac{b}{3})^{3}] &=  x^3 + bx^2 + [3x (\frac{b}{3})^{2} + (\frac{b}{3})^{3}] - [3x (\frac{b}{3})^{2} + (\frac{b}{3})^{3}] \\ 
 (x+ \frac{b}{3})^3 - [3x (\frac{b}{3})^{2} + (\frac{b}{3})^{3}] + cx + d &=  x^3 + bx^2 + [3x (\frac{b}{3})^{2} + (\frac{b}{3})^{3}] - [3x (\frac{b}{3})^{2} + (\frac{b}{3})^{3}] + cx + d \\ 
 (x+ \frac{b}{3})^3 - 3x (\frac{b}{3})^{2} - (\frac{b}{3})^{3} + cx + d &=  x^3 + bx^2 + [3x (\frac{b}{3})^{2} + (\frac{b}{3})^{3}] - [3x (\frac{b}{3})^{2} + (\frac{b}{3})^{3}] + cx + d \\ 
 (x+\frac{b}{3})^{3} - [3(\frac{b}{3})^{2} - c]x - (\frac{b}{3})^{3} + d &= x^3 + bx^2 + cx + d  \\ 
\mbox{Let }  x + \frac{b}{3} = x' \mbox{ or } x = x' - \frac{b}{3} \\
 x'^{3} - [3(\frac{b}{3})^{2} - c](x' - \frac{b}{3}) + d - (\frac{b}{3})^3 &= x^3 + bx^2 + cx + d \\
 x'^{3} - [3(\frac{b}{3})^{2} - c]x' + [3(\frac{b}{3})^{2} - c]\frac{b}{3} + d &= x^3 + bx^2 + cx + d  \\
 x'^{3} - [3(\frac{b}{3})^{2} - c]x' + 3(\frac{b}{3})^{3} - \frac{b}{3}c + d &= x^3 + bx^2 + cx + d \\
 x'^{3} +[c-3(\frac{b}{3})^2]x' + 3(\frac{b}{3})^{3} - \frac{b}{3}c + d &= x^3 + bx^2 + cx + d  \\
 \therefore A = [c-3(\frac{b}{3})^2] \quad B =  3(\frac{b}{3})^{3} - \frac{b}{3}c + d \\ 
\end{aligned}
\end{equation} 
\end{proof}

\newpage
\noindent
\[\text{Definition of topology}\]
$\text{Let }\mathcal{M} \text{ be a set. A topology }\mathcal{Q} \text{ is a subset } \mathcal{Q} \subseteq \mathcal{P}(\mathcal{M})$\\
Satisfy\\
$1. \varnothing\subseteq \mathcal{Q}, \mathcal{M} \subseteq \mathcal{Q}$\\
$2. \mathcal{U} \subseteq \mathcal{Q},   \mathcal{V} \subseteq \mathcal{Q} \implies \mathcal{U} \cap \mathcal{V} \in \mathcal{Q}$\\
$3. \mathcal{U} \in \mathcal{Q} \implies \bigcup_{\alpha \in \mathcal{A}} \mathcal{U}_\alpha \in \mathcal{Q}$\\

\noindent
\[\text{Topological Space}\]
a \textcolor{red}{topological space} is \textcolor{blue}{pair $(X, \tau)$} where $X$ is a set and $\tau$ is subset of $X$ satifying certain axioms. \textcolor{blue}{$\tau$} is called \textcolor{red}{topology}\\
1. $\emptyset$ and \textcolor{red}{space} $X$ are both in \textcolor{blue}{$\tau$}\\
2. the union of any collection of set in \textcolor{blue}{$\tau$} is contained in \textcolor{blue}{$\tau$}\\
3. the intersection of any finitly many sets in \textcolor{blue}{$\tau$} is contained in \textcolor{blue}{$\tau$}\\

\[ \text{Binomial Identity} \] 
\[  \binom{n}{k} = \binom{n}{k-1} + \binom{n-1}{k-1} \] 
\[ \binom{n}{0} = 1 \text{ with } 1 \leq k \leq n\] 

\begin{equation}
\begin{aligned}
    \text{LHS} \quad \binom{n}{k} = \frac{P(n, k)}{k!} = \frac{\frac{n!}{(n-k)!}}{k!} &= \frac{n!}{(n-k)! k!}\\
    \text{RHS} \quad \binom{n-1}{k} + \binom{n-1}{k-1} &= \frac{P(n-1, k)}{k!} + \frac{P(n-1, k-1)}{(k-1)!}  \\  
    \text{RHS} \quad \binom{n-1}{k} + \binom{n-1}{k-1} &= \frac{\frac{(n-1)!}{(n-1-k)!}}{k!} + \frac{(n-1)!}{[(n-1)-(k-1))]!(k-1)!}\\    
    \text{RHS} \quad \binom{n-1}{k} + \binom{n-1}{k-1} &= \frac{(n-1)!}{(n-k-1)!k!} + \frac{(n-1)!}{(n-k)!(k-1)!}\\    
    \text{RHS} \quad \binom{n-1}{k} + \binom{n-1}{k-1} &= \frac{(n-k)(n-1)!}{(n-k)(n-k-1)!k!} + \frac{k(n-1)!}{k(n-k)!(k-1)!}\\    
    \text{RHS} \quad \binom{n-1}{k} + \binom{n-1}{k-1} &= \frac{(n-k)(n-1)!}{(n-k)!k!} + \frac{k(n-1)!}{(n-k)!k!}\\    
    \text{RHS} \quad \binom{n-1}{k} + \binom{n-1}{k-1} &= \frac{(n-1)!(n-k+k)}{(n-k)!k!}\\    
    \text{RHS} \quad \binom{n-1}{k} + \binom{n-1}{k-1} &= \frac{(n!}{(n-k)!k!} \nonumber \quad \square\\    
\end{aligned}
\end{equation}

\newpage
\[  \binom{n+1}{k} = \binom{n}{k-1} \binom{n-1}{k-1} \] 
\includegraphics[width=10cm,scale=1]{/Users/cat/myfile/github/image/binomimage.jpg}

\newpage
\[\text{Computer Graphic Matrix} \]
\[\text{Identity}\]
\[
        \begin{bmatrix}
            1 & 0 & 0\\
            0 & 1 & 0\\   
            0 & 0 & 1  
        \end{bmatrix}
\]

\[ \text{Scalar} \]
\[
        \begin{bmatrix}
            x & 0 & 0\\
            0 & y & 0\\   
            0 & 0 & z  
        \end{bmatrix}
\]

\[ \text{Translation} \]
\[
        \begin{bmatrix}
            1 & 0 & 0 & x\\
            0 & 1 & 0 & y\\   
            0 & 0 & 1 & z\\  
            0 & 0 & 0 & 1  
        \end{bmatrix}
\]

\newpage
\[\text{Computer Graphic Matrix} \]
\[\text{Identity}\]
\[
        \begin{bmatrix}
            1 & 0 & 0\\
            0 & 1 & 0\\   
            0 & 0 & 1  
        \end{bmatrix}
\]

\[ \text{Scalar} \]
\[
        \begin{bmatrix}
            x & 0 & 0\\
            0 & y & 0\\   
            0 & 0 & z  
        \end{bmatrix}
\]

\[ \text{Translation} \]
\[
        \begin{bmatrix}
            1 & 0 & 0 & x\\
            0 & 1 & 0 & y\\   
            0 & 0 & 1 & z\\  
            0 & 0 & 0 & 1  
        \end{bmatrix}
\]

\[ \text{Rotation} \]

\begin{equation}
\begin{aligned}
    M_{z}(\beta) & =\begin{bmatrix}
            \cos\beta & -\sin\beta & 0\\
            \sin\beta & \cos\beta & 0\\   
            0      &   0    & 1  
        \end{bmatrix} \\
    M_{y}(\beta) & =\begin{bmatrix}
            \cos \beta & sin\beta & 0\\
            0      &   1    & 0    \\  
            -sin\beta & \cos\beta & 0   
        \end{bmatrix} \\
    M_{x}(\beta) & =\begin{bmatrix}
            1 &   0      &     0   \\         
            0 & \cos\beta & -\sin\beta\\
            0 & sin\beta& \cos\beta   
        \end{bmatrix} 
\end{aligned}
\end{equation}

\[ \text{Find the matrix reflects a point with respect to x-axis} \]
\[
        A \left[ 
        \begin{array}{c}
        1\\
        0    
        \end{array}
        \right]
\]
\end{document}
