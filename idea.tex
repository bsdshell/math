\documentclass[10pt]{article}
\usepackage{amsmath}
\usepackage{amsfonts}
\usepackage{amssymb}
\begin{document}
$\textbf{Definition of Monoid } $\\
$\textbf{A monoid is a triple } (A, \otimes, \overline{1})$\\
$1. \otimes \textbf{ is closed associative binary operator on the set A}$\\
$2. \overline{1} \textbf{ is identity element for } \oplus$\\
$\forall\quad a, b, c \in A$\\
$ a \otimes b  \otimes c = a \otimes (b \otimes c) $\\
$ a \otimes \overline{1} = \overline{1} \otimes a =  a$\\
\\

$\textbf{Definition of Ring}$\\
$\text{Let a, b, c }\in  \mathbb{R}$\\
$\text{There are addition and multiplication operations and satisfy associative and distributive laws}$\\
$a*b*c = a*(b*c) \textit{ and }  a*(b+c) = a*b + a*c$\\
$\text{There are additive identity } \textbf{ 0 } \text{and multiplicative identity } \textbf{ 1 }$\\
$\textbf{0} + a = a \text{ and }1*a = a$\\
$\text{There exists additive inverse } -a \text{ such that } a + (-a) = 0$\\

$\textbf{Definition of Ring}$\\
$\text{let a, b, c } \in \mathbb{R}$\\
$\text{There are two binary operations addition and multiplication and satisfy }$\\

Associative Law\\
$ a \times b \times c = a \times (b \times c) $\\

Distritutive Law\\
$a \times (b + c) = a \times b + a \times c $\\

Additive inverse\\
For all a in $\mathbb{R}$, there exists -a such that\\
$a + (-a) = 0$\\

Multiplicative identity \\
For all a in $\mathbb{R}$, there exist 1 such that\\
$1a = a$ \\




$\textbf{If }  \gcd(a, b) = 1 \textbf{ and }  a \vert bc \quad$  $\Rightarrow a \vert c$ \\
$\textbf{Proof}$ \\
$\gcd(a, b) = 1  $\\
$\Rightarrow ma+nb = 1\quad m, n \in \mathbb{N} $ \\
$\Rightarrow mac + nbc = c$ \\
$a \vert bc \Rightarrow ak = bc \quad k \in \mathbb{N} $ \\
$\Rightarrow mac + n(ak)=c \quad    (ak=bc) $ \\
$\Rightarrow a(mc + nk) = c$  \\
$\Rightarrow a \vert c $ \\
\\
$\textbf{If } \gcd(a, b) = 1 \Rightarrow ma + nb = 1 \quad m, n \in \mathbb{N}$\\
$\textbf{Proof}$\\
\\
$\textbf{Prove there is infinite prime}$\\
\\
$\textbf{Prove all the eigeivalues}\quad  \lambda \geq  0  \textbf{ if the matrix is symmetic}$\\
\\
$\textbf{If the determine of matrix } \det{A} > 0 \iff \textbf{the matrix is invertable}$\\
\\
Efficient Algorithm to compute Fibonacci Number \\
Fibonacci Sequence   
$F_{n} = F_{n} + F_{n-1}$ with $F_{0} = 0$
$F_{1} = 1$ \\
(1) Navier algorithm with recursion in $O(2^n)$ \\
(2) Use Dynamic Algorithm in $O(n)$\\
(3) Use matrix with repeated squaring to compute Fibonacci Sequence in $O(\log{n})$
\[\left(\begin{array}{cc} F_{n+1} & F_{n} \\ F_{n} & F_{n-1} \end{array} \right)^n =  \left(\begin{array}{cc}1 & 1 \\ 1 & 0 \end{array} \right)^n \]
\\
\newpage

\begin{flushleft}
Show the sum of odd number are square number\\
\medskip
1\\
1 + 3\\
1 + 3 + 5 \\
1 + 3 + 5 + ... + (2k+1) \\
\medskip
$S = \sum_{k=1}^{n} (2k-1)$ \\
$S = \sum_{k=1}^{n} 2k - \sum_{k=1}^{n} 1$ \\
$S = 2(\sum_{k=1}^{n} k) - n $ \\
$S = 2 \frac{(1+n)n}{2} - n$ \\
$S = (1+n)n - n$ \\
$S = n^2 $ \\
\end{flushleft}
composition function \\
$g \circ f \circ h $ \\
$g \circ f \colon A\to B$ \\

Find the sum of sequence of squares\\

$\sum_{k=1}^{n}((k+1)^3 - k^3) = \sum_{k=1}^{n}(k^2+1+2k)(k+1) - k^3$

$\sum_{k=1}^{n} (k^3+k+2k^2+k^2+1+2k)-k^3$

$\sum_{k=1}^{n} (k^3+3k^2+3k+1)-k^3$

$\sum_{k=1}^{n} (3k^2+3k+1)$ \\

$\sum_{k=1}^{n} ((k+1)^3-k^3) = \sum_{k=1}^{n} (k+1)^3 - \sum_{k=1}^{n} k^3$

$\Rightarrow 2^3 + 3^3 + ... + n^3 + (n+1)^3 - (1 + 2^3 + 3^3 + ... + n^3) = (n+1)^3 -1$

$\Rightarrow  \sum_{k=1}^{n}(k+1)^3 - \sum_{k=1}^{n}k^3 =(n+1)^3 - 1$

$\Rightarrow (n+1)^3-1 = \sum_{k=1}^{n}(3k^2+3k+1) $

$\Rightarrow (n+1)^3-1 = 3\sum_{k=1}^{n} k^2 +  3\sum_{k=1}^{n} k + n $

$\Rightarrow (n+1)^3-1 = 3\sum_{k=}^{n} k^2 + (n+1) n \frac{3}{2} + n$

$\Rightarrow (n+1)^3-1 -  (n+1) n \frac{3}{2} - n= 3\sum_{k=1}^{n} k^2 $

$\Rightarrow (n+1)((n+1)^2-n \frac{3}{2})-(n+1) = 3\sum_{k=1}^{n} k^2$

$\Rightarrow (n+1)( (n+1)^2 -n\frac{3}{2}-1) = 3\sum_{k=1}^{n} k^2 $	

$\Rightarrow (n+1)( n^2+1+2n-n\frac{3}{2} - 1) = 3\sum_{k=1}^{n} k^2 $	

$\Rightarrow (n+1)(n^2 + \frac{1}{2}n) = \sum_{k=1}^{n} k^2$

$\Rightarrow \frac{1}{2}(n+1)(2n^2+n)=\sum_{k=1}^{n} k^2$

$\Rightarrow \frac{1}{6}n(n+1)(2n+1) = \sum_{k=1}^{n} k^2 $

\pagebreak
Definition of Group\\
$\text{Let a, b, c} \in \mathbb{G} $\\
There is binary operation * and satisfy\\

Closure Law\\
$ a*b \in \mathbb{G} $\\

Associative Law\\
$ a*b*c = a*(b*c)$\\

Identity\\
$ \exists \mathit{e} \in \mathbb{G} \text{ such that } \mathit{e}*a = a*\mathit{e} \in \mathbb{G}$\\

Inverse\\
$ \text{If a } \in \mathbb{G}, \exists a^{-1} \in \mathbb{G} \text{ such that } a*a^{-1} = e $\\

Definition of Vector Space\\
$\text{Let }\vec{u}, \vec{v}, \vec{w} \in \vec{V} \text{ and scalars } \alpha, \beta \in \mathbb{F}$\\
Closure\\
$\vec{u} + \vec{v} \text{ and } \in \vec{V}$\\
Associative Law\\
$\vec{u} + \vec{v} + \vec{w} = \vec{u} + (\vec{v} + \vec{w})$\\
Commutative Law\\
$\vec{u} + \vec{v} = \vec{v} + \vec{u} $\\
Identity element of addition\\
$\vec{0} \in \vec{V} \text{ such that } \vec{0} + \vec{u} = \vec{u}$\\
Inverse element of addition\\
$\exists -\vec{u} \text{ such that } \vec{u} + (-\vec{u}) = \vec{0}$\\
Identity element of scalar multiplication\\
$\exists \mathit{1} \in \mathbb{F} \text{ such that } \mathit{1}\vec{u} = \vec{u}$\\
Distributivity of scale multiplication with respect to vector addition\\
$\alpha(\vec{u} + \vec{v}) = \alpha\vec{u} + \alpha\vec{v}$\\
Distributivity of scale multiplication with respect to field addition\\
$(\alpha + \beta)\vec{u} = \alpha\vec{u} + \beta\vec{u}$\\

\pagebreak
Definition of Affine Space\\
An affine space is a set of points that admits free transitive action of a vector space $\vec{V}$ That is, there is a map $X \times \vec{V} \rightarrow X:(x, \vec{v}) \mapsto x + \vec{v}$, called translation by a vector $\vec{v}$, such that\\
1. Addition of vectors corresponds to composition of translation, i.e., for all $x \in X \text{ and } \vec{u}, \vec{v} \in \vec{V}, (x + \vec{u}) + \vec{v} = x + (\vec{u} + \vec{v})$\\ 
2. The zero vector $\vec{0}$ acts as the identity vector, i.e., for all $x \in X, x + \vec{0} = x$\\
3. The action is transitive, i.e., for all $x, y \in X, \text{ exists } \vec{v} \in \vec{V} \text{ such that } y = x + \vec{v}$\\
4. The dimension of X is the dimension of vector space translations, $\vec{V}$\\\\
Or There is unique map\\
$X \times X \rightarrow \vec{V}:(x, y) \mapsto y - x \text{ such that } y = x + (y - x) \text{ for all }x, y \in X$
It furthermore satifies\\ 
1. For all $x, y, z \in X$, z - x = (z - y) + (y - x)\\
2. For all $x, y, \in X$ and $\vec{u}, \vec{v} \in \vec{V}$, $ (y + \vec{v}) - (x + \vec{u}) = (y - x) + (\vec{v} - \vec{u})$\\
3. For all $x \in X, x - x = \vec{0}$\\
4. For all $x, y \in X, y - x = -(x - y)$\\

Affine Space from linear system equation\\
Consider an $(m \times n)$ linear sytem equations\\\\
$\sum_{k=1}^{n} a_{i k} x_{k} = c_{i}, (1 \leq i \leq m) \quad\quad\quad \text{(1)}$\\\\
where $d = n - rank(M), c_{i} \ne \vec{0} \in \mathbb{R}^{m}$\\
When the system has at least one solution $x_{p}$ then the full set of solution is a d-dimension affine space
$A \subset \mathbb{R}^{n}$\\ 
Since $x_{p} \in A, \text{ we can declare point } x_{p} \text{ as origin of } A$ and then introduct A coordinates as follows:homogenous system\\

$\sum_{k=1}^{n} a_{i k} x_{k} = \vec{0} (1 \leq i \leq m)$\\\\
$\Rightarrow dim(Ker(M)) = d \quad \text{(Rank Theorem)}$\\
$\text{(1) has d-linear independent solution } \vec{b_{j}} \in \mathbb{R}^{n} \quad\quad (1 \leq j \leq d)$\\
$\text{Affine Space }\mathit{A}$ can be written as\\\\ 
$A = \Big\{ x_{p} + \sum_{j=1}^{d}\alpha_{j}\vec{b_{j}} \quad \vert \quad \alpha_{j} \in \mathbb{R} \quad\quad (1 \leq j \leq d)\Big\} $\\\\
$\text{The } \alpha_{j} \text{ can be served as coordinates in A, so that A looks as it were a d-dimension coordiate space.}$\\ 
$\text{But note that addition(+) in the space refers to the chosen point } x_{p}, \text{ and not to the origin of the base vector space}$\\

$
        \begin{bmatrix}
        1 & 2 & 3 \\
        4 & 5 & 6    
        \end{bmatrix}
        \left[
        \begin{array}{c}
        x_1 \\
        x_2 \\
        x_3 
        \end{array}
        \right] = 
        \left[ 
        \begin{array}{c}
        1 \\ 
        2 
        \end{array}
        \right]
$
\newpage

$\text{Theorem 1}$\\

The image of transformation is spanned by the image of the any basis of its domain. For $T:\vec{V} \rightarrow \vec{W}, \text{ if } \beta=\{ \vec{b_1},\vec{b_2},...,\vec{b_n} \} \text{ is a basis of }\vec{V}, 
\text{ then }T(\beta) = \{ T(\vec{b_1}), T(\vec{b_2}), ... ,T(\vec{b_n})\} \text{ spans the image of }T$\\

Proof\\
For all $\vec{v} \in \vec{V}, \vec{v} = \alpha_1\vec{b_1} + \alpha_2\vec{b_2} + ... + \alpha_n\vec{b_n}$\\
$\Rightarrow T(\vec{v}) = T(\alpha_1\vec{b_1} + \alpha_2\vec{b_2} + ... + \alpha_n\vec{b_n})$\\
$\Rightarrow T(\vec{v}) = \alpha_1 T(\vec{b_1}) + \alpha_2 T(\vec{b_2}) + ... + \alpha_n T(\vec{b_n})$\\
$\Rightarrow \{ T(\vec{b_1}), T(\vec{b_2}),...,T(\vec{b_n})\} \text{ spans the image of }T$\\

$\text{Rank Theorem}$\\
If the domain is finite dimension, then the dimension of domain is the sum of rank and nullity of the transformation\\

$\text{Let } T:\vec{V} \rightarrow \vec{W} \text{ be a linear transformation },\text{let n be the dimension of }\vec{V},$\\
$\text{let k be nullity of }T \text{ and let k be the rank of }T$\\
$\text{Show } n = k + r$\\

$\text{Let }\beta = \{ \vec{b_1}, \vec{b_2},...,\vec{b_k}\} \text{ be the basis of kernal of }T, \text{ the basis can be extended to } \gamma = \{ \vec{b_1}, \vec{b_2},...,\vec{b_k}, \vec{b_{k+1}},...,\vec{b_n}\}$\\
$\text{let }\vec{v} \in \vec{V} \Rightarrow \vec{v} = \alpha_1 \vec{b_1} + \alpha_2 + \vec{b_2} +,..., + \alpha_k \vec{b_k} + \alpha_{k+1}\vec{b}_{k+1}+,...,+\alpha_{n}\vec{b_n}$\\
$\text{Let }T(\vec{v}) = T(\alpha_1 \vec{b_1} + \alpha_2 + \vec{b_2} +,..., + \alpha_k \vec{b_k} + \alpha_{k+1}\vec{b}_{k+1}+,...,+\alpha_{n}\vec{b_n}) = \vec{0}$\\
$\Rightarrow \vec{v} = \alpha_1 \vec{b_1} + \alpha_2 + \vec{b_2} +,..., + \alpha_k \vec{b_k} + \alpha_{k+1}\vec{b}_{k+1}+,...,+\alpha_{n}\vec{b_n} \in \ker(T) \quad\quad \text{(1)}$\\
$\because \vec{v} = \sigma_1 \vec{b_1} + \sigma_2 + \vec{b_2} +,..., + \sigma_k \vec{b_k} \in \ker(T) \quad\quad \text{(2)}$\\
$(1) - (2) \Rightarrow \vec{0} = (\alpha_1-\sigma_1)\vec{b_1} + (\alpha_2 - \sigma_2)\vec{b_2}+,...,+ (\alpha_k - \sigma_k)\vec{b_k}+   \alpha_{k+1}\vec{b}_{k+1}+,...,+\alpha_{n}\vec{b_n} $\\
$\because \vec{b}_{1}, \vec{b}_{2},...,\vec{b}_{k},\vec{b}_{k+1}, \vec{b}_{k+2},...,\vec{b_n} \text{ are linearly independent}$\\
$\therefore \alpha_{k+1}, \alpha_{k+2}, ... , \alpha_{n} \text{ are all zero} \quad\quad \text{(3)}$\\
$T(\vec{v}) = T(\alpha_1 \vec{b_1}) + T(\alpha_2 \vec{b_2}) +,..., + T(\alpha_k \vec{b_k}) + T(\alpha_{k+1}\vec{b}_{k+1})+,...,+T(\alpha_{n}\vec{b_n}) = \vec{0}$\\
$T(\vec{v}) = \alpha_1 T(\vec{b_1}) + \alpha_2 T(\vec{b_2}) +,..., + \alpha_k T(\vec{b_k}) + \alpha_{k+1}T(\vec{b}_{k+1})+,...,+\alpha_{n}T(\vec{b_n}) = \vec{0}$\\
$\because \beta = \{ \vec{b_1}, \vec{b_2},...,\vec{b_k}\} \text{ is the basis of kernal of }T$\\
$\therefore T(\vec{b_1}) = \vec{0},..., T(\vec{b_k}) = \vec{0}$\\
$\therefore T(\vec{v}) = \alpha_{k+1}T(\vec{b}_{k+1})+,...,+\alpha_{n}T(\vec{b_n}) = \vec{0} \quad\quad \text{(4)}$\\
$\text{(3) and (4)} \Rightarrow \{ T(\vec{b}_{k+1}), T(\vec{b_{k+2}}), ... , T(\vec{b_{n}}) \} \text{ are linearly independent}$\\
$\Rightarrow \dim(\vec{V}) = \text{ nullity(T) } + \text{ rank(T) } \text{ or }$\\
$\Rightarrow \dim(\vec{V}) = \dim(\ker(T)) + \dim(\text{img(T)}) $\\
$\Rightarrow n = k + r \quad \square$

Affine plane\\
Affine plane is a set, whose elements are called points, and a set of subset, called lines, satifying the following three axioms:\\
1. Given two distinct points P and Q, there is one and only one containing both P and Q.\\
2. Given a line l, and a point P not in l, there is one and only one line m which is parallel to l and which passes through P.\\
3. There exists three non-collinear points.\\ 

Projective plane\\
A projective plane $\mathbb{S}$ is a set, whose elements are called points, and a set of subset, called lines, satifying the following four axioms.\\
1. Two distinct points meets P, Q of S lie on one and only one line.\\ 
2. Any two lines meet in at least one point.\\
3. There exist three non-colinear points\\
4. Every line contains at least three points.\\

\noindent
Definition of Inner Product\\

\noindent
Positivity\\
$\langle\vec{v}, \vec{v}\rangle \geq 0$\\
$\langle \vec{v} , \vec{v} \rangle = \vec{0} \iff \vec{v} = \vec{0}$\\

\noindent
Bilinearity\\
$\langle c_{1}\vec{v_1} + c_{2}\vec{v_2}, \vec{v_3}\rangle = c_{1}\langle \vec{v_1}, \vec{v_3}\rangle + c_{2}\langle\vec{v_2}, \vec{v_3} \rangle$\\

\noindent
Conjugate Symmetic\\
$\langle \vec{v_1}, \vec{v_2} \rangle = \overline{\langle \vec{v_2}, \vec{v_1} \rangle}$

\pagebreak
Calculate the Excel Sheet Row number algorithm
Latex Environment has different mode\\
{\it Math mode}\\
{\it Text mode}\\
{\it Command mode}\\



\end{document}
