\documentclass[10pt]{article}
\usepackage{amsmath}
\usepackage{amsfonts}
\usepackage{amssymb}
\usepackage{centernot}
\usepackage{xcolor}
\usepackage{graphicx}

\newcommand{\polyring}[2][Z]{\mathbb{#1}\scalebox{1}[.95]{[}#2\scalebox{1}[.95]{]}}

\begin{document}
$\textbf{Definition of Monoid } $\\
$\textbf{A monoid is a triple } (A, \otimes, \overline{1})$\\
$1. \otimes \textbf{ is closed associative binary operator on the set A}$\\
$2. \overline{1} \textbf{ is identity element for } \oplus$\\
$\forall\quad a, b, c \in A$\\
$ a \otimes b  \otimes c = a \otimes (b \otimes c) $\\
$ a \otimes \overline{1} = \overline{1} \otimes a =  a$\\
\\

$\textbf{Definition of Ring}$\\
$\text{Let a, b, c }\in  \mathbb{R}$\\
$\text{There are addition and multiplication operations and satisfy associative and distributive laws}$\\
$a*b*c = a*(b*c) \textit{ and }  a*(b+c) = a*b + a*c$\\
$\text{There are additive identity } \textbf{ 0 } \text{and multiplicative identity } \textbf{ 1 }$\\
$\textbf{0} + a = a \text{ and }1*a = a$\\
$\text{There exists additive inverse } -a \text{ such that } a + (-a) = 0$\\

$\textbf{Definition of Ring}$\\
$\text{let a, b, c } \in \mathbb{R}$\\
$\text{There are two binary operations addition and multiplication and satisfy }$\\

Associative Law\\
$ a \times b \times c = a \times (b \times c) $\\

Distritutive Law\\
$a \times (b + c) = a \times b + a \times c $\\

Additive inverse\\
For all a in $\mathbb{R}$, there exists -a such that\\
$a + (-a) = 0$\\

Multiplicative identity \\
For all a in $\mathbb{R}$, there exist 1 such that\\
$1a = a$ \\

\[\text{Group homomorphism(operation preserving)}\]
Given group $(G1, +)$ and $(G2, *)$,for all $a_1, a_2 \in G1$ and $b_1, b_2 \in G2$,\\
if $\phi(a_1 + a_2) = \phi(b_1)*\phi(b_2)$, then $\phi$ is group homomorphism\\

Given $G(\mathbb{R}, +)$ and $(\mathbb{R}, *)$, then $\phi(x) = e^x$ is homomorphism\\ 
Let $a_1, b_1 \in \mathbb{R}$ and $a_2, b_2 \in \mathbb{R}$\\ 
$\phi(a_1+b_1) = e^{a_1 + b_1}$ and $\phi(a_2)*\phi(b_2) = e^{a_2}*e^{b_2} = e^{a_2+b_2}$\\
$\Rightarrow \phi(a_1 + b_1) = \phi(a_2)*\phi(b_2)$\\
$\Rightarrow \phi(x) = e^{x}$ is homomorphism for $G(\mathbb{R}, +)$ and $G(\mathbb{R}, *)$



Ring homomorphism(operation preserving)\\
Let $\phi$ is a function between two rings $R$, then $\phi$ is a $\mathit{ring}$ homomorphism if
for all $a \in R$ and $b \in R$ 
\[\phi(a+b) = \phi(a) + \phi(b)\] 
\[\phi(ab) = \phi(a)\phi(b)\]
and \[\phi(1) = 1\]
e.g. $G(\mathbb{N}, +)$ and $G(\mathbb{Z}/\mathbb{Z}_5, +)$\\

Let $\phi: \mathbb{C} \rightarrow \mathbb{C}$ be the map send a complex number to its complex conjugate. Then $\phi$ is an automorphism of $\mathbb{C}$. 
$\phi$ is its own inverse.\\

\begin{equation}
\begin{aligned}
\phi(z) &= \overline{z}\\
\phi(z_1 + z_2) &= \overline{z_1 + z_2}\\
\overline{z_1 + z_2} &= \overline{z_1} + \overline{z_2}\\
\phi(z_1 z_2) &= \overline{z_1 z_2}\\
\overline{z_1 z_2} &= \overline{z_1} \cdot \overline{z_2} \nonumber\\
\phi(\phi(z)) &= z\\
\end{aligned}
\end{equation}

Let $\phi: \polyring{x} \rightarrow  \polyring{x}$ be the map that send $f(x)$ to $f(x+1)$. Then $\phi$ is an automorphism of $\polyring{x}$.   
The inverse map sends $f(x)$ to $f(x-1)$\\

Ideal\\
Let $R$ be a ring and let $I$ is additive subgroup of $R$, then $I$ is called an ideal of $R$ and write $I \triangleleft R$ 
\quad if $\forall a \in I$ and $\forall r \in R $, and $ ar \in I$ and $ra \in I$\\

Example\\
$R = (\mathbb{N}, +)$ and $I = (2k, +) \quad k \in \mathbb{N}$\\

Let $I$ be a kernal of $\phi$, then $I$ is an ideal of $R$\\
Let $a \in I$ and $r \in R$, then $\phi(ra) = \phi(r)\phi(a)$\\
$I$ is kernal of $\phi$ $\Rightarrow \phi(a) = 0 \therefore \phi(ra) = 0, \therefore ra \in I$\\

$\textbf{If }  \gcd(a, b) = 1 \textbf{ and }  a \vert bc \quad$  $\Rightarrow a \vert c$ \\
$\textbf{Proof}$ \\
$\gcd(a, b) = 1  $\\
$\Rightarrow ma+nb = 1\quad m, n \in \mathbb{N} $ \\
$\Rightarrow mac + nbc = c$ \\
$a \vert bc \Rightarrow ak = bc \quad k \in \mathbb{N} $ \\
$\Rightarrow mac + n(ak)=c \quad    (ak=bc) $ \\
$\Rightarrow a(mc + nk) = c$  \\
$\Rightarrow a \vert c $ \\
\\
$\textbf{If } \gcd(a, b) = 1 \Rightarrow ma + nb = 1 \quad m, n \in \mathbb{N}$\\
$\textbf{Proof}$\\
\\
$\textbf{Prove there is infinite prime}$\\
\\
$\textbf{Prove all the eigeivalues}\quad  \lambda \geq  0  \textbf{ if the matrix is symmetic}$\\
\\
$\textbf{If the determine of matrix } \det{A} > 0 \iff \textbf{the matrix is invertable}$\\
\\
Efficient Algorithm to compute Fibonacci Number \\
Fibonacci Sequence   
$F_{n} = F_{n} + F_{n-1}$ with $F_{0} = 0$
$F_{1} = 1$ \\
(1) Navier algorithm with recursion in $O(2^n)$ \\
(2) Use Dynamic Algorithm in $O(n)$\\
(3) Use matrix with repeated squaring to compute Fibonacci Sequence in $O(\log{n})$
\[\left(\begin{array}{cc} F_{n+1} & F_{n} \\ F_{n} & F_{n-1} \end{array} \right)^n =  \left(\begin{array}{cc}1 & 1 \\ 1 & 0 \end{array} \right)^n \]
\\
\newpage

\begin{flushleft}
Show the sum of odd number are square number\\
\medskip
1\\
1 + 3\\
1 + 3 + 5 \\
1 + 3 + 5 + ... + (2k+1) \\
\medskip
$S = \sum_{k=1}^{n} (2k-1)$ \\
$S = \sum_{k=1}^{n} 2k - \sum_{k=1}^{n} 1$ \\
$S = 2(\sum_{k=1}^{n} k) - n $ \\
$S = 2 \frac{(1+n)n}{2} - n$ \\
$S = (1+n)n - n$ \\
$S = n^2 $ \\
\end{flushleft}
composition function \\
$g \circ f \circ h $ \\
$g \circ f \colon A\to B$ \\

Find the sum of sequence of squares\\

$\sum_{k=1}^{n}((k+1)^3 - k^3) = \sum_{k=1}^{n}(k^2+1+2k)(k+1) - k^3$

$\sum_{k=1}^{n} (k^3+k+2k^2+k^2+1+2k)-k^3$

$\sum_{k=1}^{n} (k^3+3k^2+3k+1)-k^3$

$\sum_{k=1}^{n} (3k^2+3k+1)$ \\

$\sum_{k=1}^{n} ((k+1)^3-k^3) = \sum_{k=1}^{n} (k+1)^3 - \sum_{k=1}^{n} k^3$

$\Rightarrow 2^3 + 3^3 + ... + n^3 + (n+1)^3 - (1 + 2^3 + 3^3 + ... + n^3) = (n+1)^3 -1$

$\Rightarrow  \sum_{k=1}^{n}(k+1)^3 - \sum_{k=1}^{n}k^3 =(n+1)^3 - 1$

$\Rightarrow (n+1)^3-1 = \sum_{k=1}^{n}(3k^2+3k+1) $

$\Rightarrow (n+1)^3-1 = 3\sum_{k=1}^{n} k^2 +  3\sum_{k=1}^{n} k + n $

$\Rightarrow (n+1)^3-1 = 3\sum_{k=}^{n} k^2 + (n+1) n \frac{3}{2} + n$

$\Rightarrow (n+1)^3-1 -  (n+1) n \frac{3}{2} - n= 3\sum_{k=1}^{n} k^2 $

$\Rightarrow (n+1)((n+1)^2-n \frac{3}{2})-(n+1) = 3\sum_{k=1}^{n} k^2$

$\Rightarrow (n+1)( (n+1)^2 -n\frac{3}{2}-1) = 3\sum_{k=1}^{n} k^2 $	

$\Rightarrow (n+1)( n^2+1+2n-n\frac{3}{2} - 1) = 3\sum_{k=1}^{n} k^2 $	

$\Rightarrow (n+1)(n^2 + \frac{1}{2}n) = \sum_{k=1}^{n} k^2$

$\Rightarrow \frac{1}{2}(n+1)(2n^2+n)=\sum_{k=1}^{n} k^2$

$\Rightarrow \frac{1}{6}n(n+1)(2n+1) = \sum_{k=1}^{n} k^2 $

\pagebreak
Definition of Group\\
$\text{Let a, b, c} \in \mathbb{G} $\\
There is binary operation * and satisfy\\

Closure Law\\
$ a*b \in \mathbb{G} $\\

Associative Law\\
$ a*b*c = a*(b*c)$\\

Identity\\
$ \exists \mathit{e} \in \mathbb{G} \text{ such that } \mathit{e}*a = a*\mathit{e} \in \mathbb{G}$\\

Inverse\\
$ \text{If a } \in \mathbb{G}, \exists a^{-1} \in \mathbb{G} \text{ such that } a*a^{-1} = e $\\

Definition of Vector Space\\
$\text{Let }\vec{u}, \vec{v}, \vec{w} \in \vec{V} \text{ and scalars } \alpha, \beta \in \mathbb{F}$\\
Closure\\
$\vec{u} + \vec{v} \text{ and } \in \vec{V}$\\
Associative Law\\
$\vec{u} + \vec{v} + \vec{w} = \vec{u} + (\vec{v} + \vec{w})$\\
Commutative Law\\
$\vec{u} + \vec{v} = \vec{v} + \vec{u} $\\
Identity element of addition\\
$\vec{0} \in \vec{V} \text{ such that } \vec{0} + \vec{u} = \vec{u}$\\
Inverse element of addition\\
$\exists -\vec{u} \text{ such that } \vec{u} + (-\vec{u}) = \vec{0}$\\
Identity element of scalar multiplication\\
$\exists \mathit{1} \in \mathbb{F} \text{ such that } \mathit{1}\vec{u} = \vec{u}$\\
Distributivity of scale multiplication with respect to vector addition\\
$\alpha(\vec{u} + \vec{v}) = \alpha\vec{u} + \alpha\vec{v}$\\
Distributivity of scale multiplication with respect to field addition\\
$(\alpha + \beta)\vec{u} = \alpha\vec{u} + \beta\vec{u}$\\

\pagebreak
Definition of Affine Space\\
An affine space is a set of points that admits free transitive action of a vector space $\vec{V}$ That is, there is a map $X \times \vec{V} \rightarrow X:(x, \vec{v}) \mapsto x + \vec{v}$, called translation by a vector $\vec{v}$, such that\\
1. Addition of vectors corresponds to composition of translation, i.e., for all $x \in X \text{ and } \vec{u}, \vec{v} \in \vec{V}, (x + \vec{u}) + \vec{v} = x + (\vec{u} + \vec{v})$\\ 
2. The zero vector $\vec{0}$ acts as the identity vector, i.e., for all $x \in X, x + \vec{0} = x$\\
3. The action is transitive, i.e., for all $x, y \in X, \text{ exists } \vec{v} \in \vec{V} \text{ such that } y = x + \vec{v}$\\
4. The dimension of X is the dimension of vector space translations, $\vec{V}$\\\\
Or There is unique map\\
$X \times X \rightarrow \vec{V}:(x, y) \mapsto y - x \text{ such that } y = x + (y - x) \text{ for all }x, y \in X$
It furthermore satifies\\ 
1. For all $x, y, z \in X$, z - x = (z - y) + (y - x)\\
2. For all $x, y, \in X$ and $\vec{u}, \vec{v} \in \vec{V}$, $ (y + \vec{v}) - (x + \vec{u}) = (y - x) + (\vec{v} - \vec{u})$\\
3. For all $x \in X, x - x = \vec{0}$\\
4. For all $x, y \in X, y - x = -(x - y)$\\

Affine Space from linear system equation\\
Consider an $(m \times n)$ linear sytem equations\\\\
$\sum_{k=1}^{n} a_{i k} x_{k} = c_{i}, (1 \leq i \leq m) \quad\quad\quad \text{(1)}$\\\\
where $d = n - rank(M), c_{i} \ne \vec{0} \in \mathbb{R}^{m}$\\
When the system has at least one solution $x_{p}$ then the full set of solution is a d-dimension affine space
$A \subset \mathbb{R}^{n}$\\ 
Since $x_{p} \in A, \text{ we can declare point } x_{p} \text{ as origin of } A$ and then introduct A coordinates as follows:homogenous system\\

$\sum_{k=1}^{n} a_{i k} x_{k} = \vec{0} (1 \leq i \leq m)$\\\\
$\Rightarrow dim(Ker(M)) = d \quad \text{(Rank Theorem)}$\\
$\text{(1) has d-linear independent solution } \vec{b_{j}} \in \mathbb{R}^{n} \quad\quad (1 \leq j \leq d)$\\
$\text{Affine Space }\mathit{A}$ can be written as\\\\ 
$A = \Big\{ x_{p} + \sum_{j=1}^{d}\alpha_{j}\vec{b_{j}} \quad \mid \quad \alpha_{j} \in \mathbb{R} \quad\quad (1 \leq j \leq d)\Big\} $\\\\
$\text{The } \alpha_{j} \text{ can be served as coordinates in A, so that A looks as it were a d-dimension coordiate space.}$\\ 
$\text{But note that addition(+) in the space refers to the chosen point } x_{p}, \text{ and not to the origin of the base vector space}$\\

$
        \begin{bmatrix}
        1 & 2 & 3 \\
        4 & 5 & 6    
        \end{bmatrix}
        \left[
        \begin{array}{c}
        x_1 \\
        x_2 \\
        x_3 
        \end{array}
        \right] = 
        \left[ 
        \begin{array}{c}
        1 \\ 
        2 
        \end{array}
        \right]
$
\newpage

\noindent
Theorem 1\\
The image of transformation is spanned by the image of the any basis of its domain. For $T:\vec{V} \rightarrow \vec{W}, \text{ if } \beta=\{ \vec{b_1},\vec{b_2},...,\vec{b_n} \} \text{ is a basis of }\vec{V}, 
\text{ then }T(\beta) = \{ T(\vec{b_1}), T(\vec{b_2}), ... ,T(\vec{b_n})\} \text{ spans the image of }T$\\

\noindent
Proof\\
For all $\vec{v} \in \vec{V}, \vec{v} = \alpha_1\vec{b_1} + \alpha_2\vec{b_2} + ... + \alpha_n\vec{b_n}$\\
$\Rightarrow T(\vec{v}) = T(\alpha_1\vec{b_1} + \alpha_2\vec{b_2} + ... + \alpha_n\vec{b_n})$\\
$\Rightarrow T(\vec{v}) = \alpha_1 T(\vec{b_1}) + \alpha_2 T(\vec{b_2}) + ... + \alpha_n T(\vec{b_n})$\\
$\Rightarrow \{ T(\vec{b_1}), T(\vec{b_2}),...,T(\vec{b_n})\} \text{ spans the image of }T$\\

\noindent
Rank Theorem\\
If the domain is finite dimension, then the dimension of domain is the sum of rank and nullity of the transformation\\
$\text{Let } T:\vec{V} \rightarrow \vec{W} \text{ be a linear transformation },\text{let n be the dimension of }\vec{V},$\\
$\text{let k be nullity of }T \text{ and let k be the rank of }T$\\
$\text{Show } n = k + r$\\

$\text{Let }\beta = \{ \vec{b_1}, \vec{b_2},...,\vec{b_k}\} \text{ be the basis of kernal of }T, \text{ the basis can be extended to } \gamma = \{ \vec{b_1}, \vec{b_2},...,\vec{b_k}, \vec{b_{k+1}},...,\vec{b_n}\}$\\
$\text{let }\vec{v} \in \vec{V} \Rightarrow \vec{v} = \alpha_1 \vec{b_1} + \alpha_2 + \vec{b_2} +,..., + \alpha_k \vec{b_k} + \alpha_{k+1}\vec{b}_{k+1}+,...,+\alpha_{n}\vec{b_n}$\\
$\text{Let }T(\vec{v}) = T(\alpha_1 \vec{b_1} + \alpha_2 + \vec{b_2} +,..., + \alpha_k \vec{b_k} + \alpha_{k+1}\vec{b}_{k+1}+,...,+\alpha_{n}\vec{b_n}) = \vec{0}$\\
$\Rightarrow \vec{v} = \alpha_1 \vec{b_1} + \alpha_2 + \vec{b_2} +,..., + \alpha_k \vec{b_k} + \alpha_{k+1}\vec{b}_{k+1}+,...,+\alpha_{n}\vec{b_n} \in \ker(T) \quad\quad \text{(1)}$\\
$\because \vec{v} = \sigma_1 \vec{b_1} + \sigma_2 + \vec{b_2} +,..., + \sigma_k \vec{b_k} \in \ker(T) \quad\quad \text{(2)}$\\
$(1) - (2) \Rightarrow \vec{0} = (\alpha_1-\sigma_1)\vec{b_1} + (\alpha_2 - \sigma_2)\vec{b_2}+,...,+ (\alpha_k - \sigma_k)\vec{b_k}+   \alpha_{k+1}\vec{b}_{k+1}+,...,+\alpha_{n}\vec{b_n} $\\
$\because \vec{b}_{1}, \vec{b}_{2},...,\vec{b}_{k},\vec{b}_{k+1}, \vec{b}_{k+2},...,\vec{b_n} \text{ are linearly independent}$\\
$\therefore \alpha_{k+1}, \alpha_{k+2}, ... , \alpha_{n} \text{ are all zero} \quad\quad \text{(3)}$\\
$T(\vec{v}) = T(\alpha_1 \vec{b_1}) + T(\alpha_2 \vec{b_2}) +,..., + T(\alpha_k \vec{b_k}) + T(\alpha_{k+1}\vec{b}_{k+1})+,...,+T(\alpha_{n}\vec{b_n}) = \vec{0}$\\
$T(\vec{v}) = \alpha_1 T(\vec{b_1}) + \alpha_2 T(\vec{b_2}) +,..., + \alpha_k T(\vec{b_k}) + \alpha_{k+1}T(\vec{b}_{k+1})+,...,+\alpha_{n}T(\vec{b_n}) = \vec{0}$\\
$\because \beta = \{ \vec{b_1}, \vec{b_2},...,\vec{b_k}\} \text{ is the basis of kernal of }T$\\
$\therefore T(\vec{b_1}) = \vec{0},..., T(\vec{b_k}) = \vec{0}$\\
$\therefore T(\vec{v}) = \alpha_{k+1}T(\vec{b}_{k+1})+,...,+\alpha_{n}T(\vec{b_n}) = \vec{0} \quad\quad \text{(4)}$\\
$\text{(3) and (4)} \Rightarrow \{ T(\vec{b}_{k+1}), T(\vec{b_{k+2}}), ... , T(\vec{b_{n}}) \} \text{ are linearly independent}$\\
$\Rightarrow \dim(\vec{V}) = \text{ nullity(T) } + \text{ rank(T) } \text{ or }$\\
$\Rightarrow \dim(\vec{V}) = \dim(\ker(T)) + \dim(\text{img(T)}) $\\
$\Rightarrow n = k + r \quad \square$\\

\noindent
Affine plane\\
Affine plane is a set, whose elements are called points, and a set of subset, called lines, satifying the following three axioms:\\
1. Given two distinct points P and Q, there is one and only one containing both P and Q.\\
2. Given a line l, and a point P not in l, there is one and only one line m which is parallel to l and which passes through P.\\
3. There exists three non-collinear points.\\ 

\noindent
A coordinate system in an affine space $(\mathbf{X}, \mathbf{V}) \text{ consists of a point } \mathbf{O} \in \mathbf{X}$ is called origin, and basis 
$\vec{v_1},...,\vec{v_n}\quad\text{ for } \vec{V} \text{ Any point }\mathbf{x} \in \mathbf{X} \text{ can be written as}$\\

$\mathbf{x} - \mathbf{O} = \mathbf{x} - \mathbf{O}$\\
$\Rightarrow \mathbf{x} = \mathbf{O} + (\mathbf{x} - \mathbf{O}) = \mathbf{O} + \sum_{k=1}^{n}\mathit{x_k}\vec{v_k}$\\
$\text{where the numbers }\mathit{x_1},...,\mathit{x_n}\text{ are the coordinates for vector }\mathbf{x} - \mathbf{O}\text{ with respect to the basis }\vec{v_1},...\vec{v_n}.$\\
$\text{They are now also called the coordinates for }\mathbf{x} \text{ with respect to the coordinate system }\mathbf{O}, \vec{v_1},...,\vec{v_n}$\\

\noindent
Projective plane\\
A projective plane $\mathbb{S}$ is a set, whose elements are called points, and a set of subset, called lines, satifying the following four axioms.\\
1. Two distinct points meets P, Q of S lie on one and only one line.\\ 
2. Any two lines meet in at least one point.\\
3. There exist three non-colinear points\\
4. Every line contains at least three points.\\

\noindent
Manifold\\
$\text{An subset }\mathbf{S} \text{ of } \mathbb{R}^{m} \text{ is called a manifold of dimension of d if every point p of }\mathbf{S} \text{ has a neighbourhood in }$
$\mathbf{S}\text{ which is homeomorphic to an open set of }\mathbb{R}^{d}$\\

\noindent
Homeomorphrism\\
$\text{Let S be a subset of }\mathbb{R}^{m} \text{ and sp be the subset of }\mathbb{R}^{n}. \text{ A map } \mathit{f}: \mathbb{R}^{m} \mapsto \mathbb{R}^{n}$\\
$\text{ is called homeomorphism if }\mathit{f} \text{ is continuous and bijective and }\mathit{f}^{-1} \text{ is continuous}$\\


\noindent
Definition Open and Close Sets\\
1. $\mathbf{S}$ is said to be open set if every point of $\mathbf{S}$ is an interior of $\mathbf{S}$\\
2. $\mathbf{S}$ is said to be closed set if $\mathbb{R}\setminus\mathbf{S}$ is open\\  

\noindent
Proposition\\
1. $\mathbf{S} \text{ is open if there exists } \delta > 0 \text{ such that } (\mathit{s} - \delta, \mathit{s} + \delta) \subseteq \mathbf{S}$\\
2. $\mathbf{S} \text{ is open if any }\mathit{s} \in \mathbf{S} \text{ there exists a neighbourhood of }\mathit{s} \text{ included in }\mathbf{S}$\\  


\noindent
Terminology\\
$\mathcal{M} \text{ Set (ZFC) book}$\\
$\mathcal{Q} \text{ topology =: set of open set}$\\
$(\mathcal{M}, \mathcal{Q}) \text{ toplogy space}$\\
$\mathcal{U} \in \mathcal{Q} \iff \text{ all } \mathcal{U} \subseteq \mathcal{M} \text{ and open set}$\\
$\mathcal{M} \setminus \mathcal{A} \iff   \text{ all }    \mathcal{A} \subseteq \mathcal{M} \text{ closed set}$\\
$\text{open } \centernot\implies \text{ closed}$\\
$\text{open } \centernot\impliedby \text{ closed}$\\

\noindent
Definition of Inner Product
\noindent
Positivity\\
$\langle\vec{v}, \vec{v}\rangle \geq 0$\\
$\langle \vec{v} , \vec{v} \rangle = \vec{0} \iff \vec{v} = \vec{0}$\\

\noindent
Bilinearity\\
$\langle c_{1}\vec{v_1} + c_{2}\vec{v_2}, \vec{v_3}\rangle = c_{1}\langle \vec{v_1}, \vec{v_3}\rangle + c_{2}\langle\vec{v_2}, \vec{v_3} \rangle$\\

\noindent
Conjugate Symmetic\\
$\langle \vec{v_1}, \vec{v_2} \rangle = \overline{\langle \vec{v_2}, \vec{v_1} \rangle}$


\noindent
Proof  Cauchy-Schwarz Inequality by picture\\
$\vert a \cdot b \vert \leq \lVert a \lVert \lVert b \lVert$


\pagebreak
Calculate the Excel Sheet Row number algorithm
Latex Environment has different mode\\
{\it Math mode}\\
{\it Text mode}\\
{\it Command mode}\\

Elliptic Curve and Group Structure Conic Curve\\
$
    \begin{bmatrix}
    x & y & z 
    \end{bmatrix}
    \begin{bmatrix}
    a & b & c \\
    d & e & f \\   
    g & h & i \\   
    \end{bmatrix}
    \left[
    \begin{array}{c}
    x \\
    y \\
    z 
    \end{array}
    \right] = 
    \left[ 
    \begin{array}{c}
    0  
    \end{array}
    \right]
$\\
$ ax^{2} + ey^{2} + iz^{2} + (b+d)xy + (c+g)xz + (f+h)yz = 0$\\
\\
$\mathit{(x, y, z)} = \mathit{(x/z, y/z, 1)} (z \neq 0)$\\ 
\\

\noindent
$
\begin{array}{lcl}
ax^{2} + ey^{2} + (b+d)xy + (c+g)x + (f+h)y + i &=& 0\\
ax + by + c &=& 0
\end{array}
$\\
sub (1) into (2) 
$\Rightarrow ax^{2} + bx + c = 0$\\
$\Rightarrow x = \frac{-b \pm \sqrt{b^{2} - 4ac}}{2a}$
\\

Exponential backoff algorithm\\
\\
$\frac{1}{N+1} \sum_{i=1}^{k}i$\\
For example, the expected backoff time for the third collision, one could 
calculate the maximum backoff time, N\\
$N = 2^{c} - 1 (c = 3)$
$N = 7$\\
Calculate the mean of backoff time for the third collision(c=3)\\
$\mathbf{E(c)} = \frac{1}{N+1}\sum_{i=0}^{N} i$\\
$\mathbf{E(c)} = \frac{1}{N+1}\sum_{i=0}^{N} \Rightarrow \frac{1}{N+1} \frac{N(N+1)}{2} = \frac{N}{2}$\\
$\mathbf{E(3)} = \frac{1}{7+1}\sum_{i=0}^{N} i = \frac{1}{8}(0+1+2+3+4+5+6+7) = \frac{28}{8}$\\
$\mathbf{E(3)} = 3.5$\\

Prove Square root of two is irrational
$\sqrt{2} \notin \mathbb{Q}$\\
$\text{Assume } \sqrt{2} \in \mathbb{Q}$\\
$\text{let }n = min\{ n \in \mathbb{N} \mid n*\sqrt{2} \in \mathbb{N}\}$\\
$\Rightarrow n*(\sqrt{2} - 1)*\sqrt{2} \in \mathbb{Q}$\\
$\because \sqrt{2} - 1 < 1$\\
$\Rightarrow n*(\sqrt{2} - 1)*\sqrt{2} < n*\sqrt{2}$\\
$\Rightarrow n*(\sqrt{2} - 1) < n \text{ such as } n*(\sqrt{2} - 1)*\sqrt{2} \in \mathbb{N}$\\
$\Rightarrow \text{This contracts our assumption } \quad \square$\\

\pagebreak
\noindent
\[ \text{Prove Square root of 2 is irrational} \text{ [ Geometric proof ] }\]
$\text{Assume } \sqrt{2} \in \mathbb{Q}$\\
$\Rightarrow \frac{a}{b} = \sqrt{2} \quad a, b \in \mathbb{N} \text{ and } \gcd(a, b) = 1, a > b$\\

given a right issoseles triangle\\
$AB = AC = AE, \quad FE \text{  tangles to arc at point } E $\\
$\Rightarrow AF = EF$\\
$\text{Let } AB = AC = 1$\\
$\Rightarrow BC = \sqrt{2}$\\
$\Rightarrow FB = 1-EB = 1-(\sqrt{2} - 1) = 2-\sqrt{2}$\\
$\because \frac{AC}{CB} = \frac{EB}{FB} = \frac{1}{\sqrt{2}} = \frac{\sqrt{2}-1}{2-\sqrt{2}}$\\
$\therefore \frac{1}{\sqrt{2}} =\frac{1}{\frac{a}{b}} = \frac{\sqrt{2}-1}{2-\sqrt{2}} = \frac{\frac{a}{b} - 1}{2-\frac{a}{b}}$\\
$\therefore \frac{b}{a} = \frac{ \frac{a-b}{b} } {\frac{2b - a}{b}} = \frac{a-b}{2b-a}$\\
$\because \sqrt{2} < \sqrt{4} = 2 \therefore  a < 2b $\\
$\Rightarrow  \color{red} a-b < b $\\
$\because 2a > 2b$\\
$\Rightarrow  \color{red}2b -a < a$\\
$\text{That contracts our assumption } \gcd(a, b) = 1$\\
$\Rightarrow \frac{a}{b} \notin \mathbb{Q}$

\pagebreak
\noindent
Geometric proof: square root of two is irrational\\
$\text{Given an isosceles right triangle from above and let } \gcd(a, b)=1, \text{ from Pythagorean theorem}$\\
$\Rightarrow a^2 = b^2 + b^2$\\
$\Rightarrow \sqrt{2} = \frac{a}{b}$\\

singular point on affine plane curve\\
$\text{If } p \in (x, y) \text{ and } \frac{df}{dx}, \frac{df}{dy} \text{ are undefined on } p \in (x, y), \text{ then } p\in (x, y) \text{ is singular point}$\\


\[\text{Eisenstein series}\]
$L = [w_1, w_2] \in \mathbb{C}, G(L) = \sum_{w\in L\setminus \{0, 0\}} \frac{1}{w^k}$\\
$\text{Let lattices } L=[1, \tau] \text{ and parametrized by }\tau  \text{ in the upper half plane } \mathbb{H} = \{ z \in\mathbb{C} : \Im(z) > 0 \} $\\
$G(L) = G([1, \tau]) =G(\tau) = \sum_{m, n \in \mathbb{Z}}' \frac{1}{(m+n\tau)^k}$\\

\noindent
\[\text{Show } G_{k}(\tau + 1) = G_{k}(\tau)\]
$G_{k}(\tau + 1) = \sum_{m, n \in \mathbb{Z}}' \frac{1}{(m+n(\tau+1))^k} = \sum_{m, n \in \mathbb{Z}}' \frac{1}{(m + n + n\tau)^k} = \sum_{m', n \in \mathbb{Z} }' \frac{1}{(m' + n\tau)^k}$\\
$\Rightarrow G_{k}(\tau + 1) = G_{k}(\tau)$\\

\noindent
\[\text{Show } G_{k}(\tau) = \tau^{-k} G_{k}(\frac{-1}{\tau})\]
$\tau^{-k} G_{k}(\frac{-1}{\tau}) =  \tau^{-k} \sum_{(m, n \in \mathbb{Z} )}' \frac{1}{(m+\frac{n}{\tau})^{k}}$\\
$\tau^{-k} \sum_{m, n \in \mathbb{Z}}' \frac{1}{(m + \frac{n}{\tau})^{k}} = \tau^{-k} \sum_{m, n \in \mathbb{Z}}' \frac{1}{(\frac{m\tau}{\tau} + \frac{n}{\tau})^{k}} = \sum_{m, n \in \mathbb{Z}}' \textcolor{red}{(\tau^{-k}\tau^{k})} (m\tau + n)^{-k})$\\
$\Rightarrow G_{k}(\tau) = \tau^{-k} G_{k}(\frac{-1}{\tau})$\\


\noindent
\[\text{Show }G_{k}(\tau) = 0 \text{ if } k = (2j+1) \quad j \in \mathbb{Z} \]
$\text{For each }\omega = (m+n\tau) \in L, \text{there exists } -\omega = -(m+n\tau) \in L $\\
$\because \omega ^{-(2J+1)} + (-1)^{-(2J+1)}\omega^{-(2j+1)} = 0$\\
$\therefore \sum_{m,n \in \mathbb{Z}}' (m+n\tau)^{-k} = 0\quad \text{ for all } k \in (2j+1)$\\

\noindent
\[\text{Show for any lattices L, the sum } \sum_{\omega \in \mathit{L}}' \frac{1}{\omega^{k}} \text{ converges absoluately for all } k > 2 \]
Proof:


\pagebreak
\noindent
\[\text{Show }\frac{1}{(1-x)^2} = \sum_{n=0}^{\infty} (n+1)x^n \quad |x| < 1\]
Proof:
$\text{For all } |x| < 1, \text{ power series expansion} $\\
\[(\frac{1}{1-x}\big) = (\sum_{n=0}^{\infty} x^n)\]\\
Differentiate both sides\\
\begin{equation}
\begin{aligned}
(\frac{1}{1-x}\big)' &= (\sum_{n=0}^{\infty} x^n)' \\
(\frac{1}{1-x}\big)' &= (1-x)^{-2}\\ 
\sum_{n=1}^{\infty} nx^{n-1} &= \sum_{i=0}^{\infty} (i+1)x^{i} \quad (n-1=i)\\
\frac{1}{(1-x)^2} &= \sum_{n=0}^{\infty} (n+1)x^{n} \quad \text{ sub }(i = n)\\
\end{aligned}
\end{equation}

\noindent
\[\text{Weierstrass function } \wp \text{-function of lattice L is defined by} \]
$\wp(z) = \wp(z; L) = \frac{1}{z^2} + \sum_{w \in L}' \big[ \frac{1}{(z-\omega)^2} + \frac{1}{\omega^2} \big]$\\

\noindent
Holomorphic\\
A function $f(z)$ defined on some open neibourhood of a point $z_{0} \in \mathbb{C}$ is said to be holomorphic at $z_{0}$ if the complex derivative 
\[ f'(z_{0}) = \lim_{h \to 0 } \frac{f(z_{0} + h) - f(z_{0})}{h} \] exists.
We said $f$ is holomorphic on an open set $\Omega$ if it is holomorphic at every $z_{0} \in \Omega$ and we said $f$ is holomorphic in a closed set $\mathbf{C}$ if it is holomorphic on some open set $\Omega$ containing $\mathbf{C}$.
Functions are holomorphic on all of $\mathbb{C}$ are said to be $\mathit{entire}$\\

\noindent
\[ \text{Show } f(z) = z^2 \text{ is holomorphic} \]
\begin{equation}
\begin{aligned}
f'(z) &= \lim_{h \to 0} \frac{(z+h)^2 - z^2}{h} \\
f'(z) &= \lim_{h \to 0} \frac{z^2 + 2hz + h^2 - z^2}{h}\\
f'(z) &= \lim_{h \to 0} \frac{2hz + h^2}{h} \nonumber \\
f'(z) &= \lim_{h \to 0} 2z + h\\
f'(z) &= 2z
\end{aligned}
\end{equation}

\pagebreak
\noindent
Differential\\
A function $f(x)$ is differential on $x_{0} \in \mathbb{R}$ if \\
\[  f'(x_{0}) =  \lim_{h \to 0} \frac{f(x_{0} +h) - f(x_{0})}{h} \]
exists\\

\[\text{Show } f(x) = x^{2} \text{ is differentiable for all } x  \in \mathbb{R} \]
\begin{equation}
\begin{aligned}
f'(x) &=  \lim_{h \to 0} \frac{f(x +h) - f(x)}{h}\\
f'(x) &= \lim_{h \to 0} \frac{x^2 + h^2 + 2xh - x^2  }{h} \nonumber \\
f'(x) &= \lim_{h \to 0} \frac{h^2 + 2xh}{h}\\
f'(x) &= \lim_{h \to 0} h+2x\\     
f'(x) &= 2x \quad \text{ for all } x \in \mathbb{R}
\end{aligned}
\end{equation}

\pagebreak
\noindent
\[ \text{Elliptic Curve}\]\\
Find the formula for \[ s(n) = \sum_{k=1}^{n} k^2 \]\\
\[ s(n) = \frac{(2n+1)(n+1)n}{2 \times 3} \]\\
Let $y^2 = S(n)$ and $x=n$\\
\[ y^2 = \frac{1}{6}x(x+1)(2x+1) \]\\
\includegraphics[scale=4]{/Users/cat/myfile/github/geogebra/elliptic1.png}

$ (x, y)=(0, 0), (-1, 0),(-\frac{1}{2}, 0)$ are on the curve\\

\noindent
\[ y^2 = x^3 + bx^2 + cx + d \]
$\because (x+ \frac{b}{3})^3 =  x^3 + 3x^2 \frac{b}{3} + 3x (\frac{b}{3})^{2} + (\frac{b}{3})^{3}$\\
$y^2 = (x+ \frac{b}{3})^3  - 3x (\frac{b}{3})^{2} - (\frac{b}{3})^{3} + cx + d$\\
$y^2 = (x+\frac{b}{3})^{3} - [3(\frac{b}{3})^{2} - c]x + d$\\
let x + $\frac{b}{3} = x'$\\
$y^2 = x'^{3} - [3(\frac{b}{3})^{2} - c][x' - \frac{b}{3}] + d$\\
$y^2 = x'^{3} - [3(\frac{b}{3})^{2} - c]x' + 3(\frac{b}{3})^{3} - \frac{b}{3}c + d $\\
$y^2 = x'^{3} +[c-3(\frac{b}{3})^2]x' + 3(\frac{b}{3})^{3} - \frac{b}{3}c + d $

\noindent
\[ \text{For any lattice } L, \text{the sum of  }\sum_{\omega_k \in L}'^{\infty} \frac{1}{\omega_k} \text{ is converges absoluately for all } k > 2 \]

\newpage
\noindent
\[\text{Definition of topology}\]
$\text{Let }\mathcal{M} \text{ be a set. A topology }\mathcal{Q} \text{ is a subset } \mathcal{Q} \subseteq \mathcal{P}(\mathcal{M})$\\
Satisfy\\
$1. \varnothing\subseteq \mathcal{Q}, \mathcal{M} \subseteq \mathcal{Q}$\\
$2. \mathcal{U} \subseteq \mathcal{Q},   \mathcal{V} \subseteq \mathcal{Q} \implies \mathcal{U} \cap \mathcal{V} \in \mathcal{Q}$\\
$3. \mathcal{U} \in \mathcal{Q} \implies \bigcup_{\alpha \in \mathcal{A}} \mathcal{U}_\alpha \in \mathcal{Q}$\\

\noindent
\[\text{Topological Space}\]
a \textcolor{red}{topological space} is \textcolor{blue}{pair $(X, \tau)$} where $X$ is a set and $\tau$ is subset of $X$ satifying certain axioms. \textcolor{blue}{$\tau$} is called \textcolor{red}{topology}\\
1. $\emptyset$ and \textcolor{red}{space} $X$ are both in \textcolor{blue}{$\tau$}\\
2. the union of any collection of set in \textcolor{blue}{$\tau$} is contained in \textcolor{blue}{$\tau$}\\
3. the intersection of any finitly many sets in \textcolor{blue}{$\tau$} is contained in \textcolor{blue}{$\tau$}

\end{document}
