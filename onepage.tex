\documentclass[10pt]{article}
\usepackage{tipa}
\usepackage{pagecolor,lipsum}
\usepackage{amsmath}
\usepackage{amsfonts}
\usepackage{amssymb}
\usepackage{centernot}
\usepackage{xcolor}
\usepackage{graphicx}
\begin{document}
\noindent
\[ \text{Binomial Identity} \] 
\[  \binom{n}{k} = \binom{n}{k-1} \binom{n-1}{k-1} \] 
\[ \binom{n}{0} = 1 \text{ with } 1 \leq k \leq n\] 


\begin{equation}
\begin{aligned}
\text{LHS} \quad \binom(n, k) = \frac{P(n, k)}{k!} = \frac{\frac{n!}{(n-k)!}}{k!} &= \frac{n!}{(n-k)! k!}\\
    \text{RHS} \quad \binom{n-1}{k} + \binom{n-1}{k-1} &= \frac{P(n-1, k)}{k!} + \frac{P(n-1, k-1)}{(k-1)!}  \\  
    \text{RHS} \quad \binom{n-1}{k} + \binom{n-1}{k-1} &= \frac{\frac{(n-1)!}{(n-1-k)!}}{k!} + \frac{(n-1)!}{[(n-1)-(k-1))]!(k-1)!}\\    
    \text{RHS} \quad \binom{n-1}{k} + \binom{n-1}{k-1} &= \frac{(n-1)!}{(n-k-1)!k!} + \frac{(n-1)!}{(n-k)!(k-1)!}\\    
    \text{RHS} \quad \binom{n-1}{k} + \binom{n-1}{k-1} &= \frac{(n-k)(n-1)!}{(n-k)(n-k-1)!k!} + \frac{k(n-1)!}{k(n-k)!(k-1)!}\\    
    \text{RHS} \quad \binom{n-1}{k} + \binom{n-1}{k-1} &= \frac{(n-k)(n-1)!}{(n-k)!k!} + \frac{k(n-1)!}{(n-k)!k!}\\    
    \text{RHS} \quad \binom{n-1}{k} + \binom{n-1}{k-1} &= \frac{(n-1)!(n-k+k)}{(n-k)!k!}\\    
    \text{RHS} \quad \binom{n-1}{k} + \binom{n-1}{k-1} &= \frac{(n!}{(n-k)!k!} \nonumber \quad \square\\    
\end{aligned}
\end{equation}

\[ \text{Division Ring} \] 
Division Ring is a set $F$, together with two operations + and $\times$ 
1. $F$ is abelian group under +
2. The non-zero elements of $F$ form group under $\times$ (not necessary commutative)


\newpage
\[  \binom{n+1}{k} = \binom{n}{k-1} \binom{n-1}{k-1} \] 
\includegraphics[width=10cm,scale=1]{/Users/cat/myfile/github/image/binomimage.jpg}\\
\end{document}
